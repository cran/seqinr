\documentclass{article}
\usepackage{graphicx} % Extended graphics inclusions
\usepackage{float}
\usepackage{url} % For \url{}
\usepackage{../config/atxy} % For front cover
\usepackage{amsfonts} % Needed for some fonts
\usepackage[usenames]{color} % Needed for colored R input/output
\usepackage{pdfcolmk} % Correct some problems with the color stack


\title{Installation of a local ACNUC socket server and of a local ACNUC database on your machine.}
\author{Penel, S.}

\usepackage{Sweave}
\begin{document}
%
% To change the R input/output style:
%
\definecolor{Soutput}{rgb}{0,0,0.56}
\definecolor{Sinput}{rgb}{0.56,0,0}
\DefineVerbatimEnvironment{Sinput}{Verbatim}
{formatcom={\color{Sinput}},fontsize=\footnotesize, baselinestretch=0.75}
\DefineVerbatimEnvironment{Soutput}{Verbatim}
{formatcom={\color{Soutput}},fontsize=\footnotesize, baselinestretch=0.75}
%
% This removes the extra spacing after code and output chunks in Sweave,
% but keeps the spacing around the whole block.
%
\fvset{listparameters={\setlength{\topsep}{0pt}}}
\renewenvironment{Schunk}{\vspace{\topsep}}{\vspace{\topsep}}
%
% Rlogo
%
\newcommand{\Rlogo}{\protect\includegraphics[height=1.8ex,keepaspectratio]{../figs/Rlogo.pdf}}
%
% Shortcut for seqinR:
%
\newcommand{\seqinr}{\texttt{seqin\bf{R}}}
\newcommand{\Seqinr}{\texttt{Seqin\bf{R}}}
\fvset{fontsize= \scriptsize}
%
% R output options and libraries to be loaded.
%
%
%  Sweave Options
%
% Put all figures in the fig folder and start the name with current file name.
% Do not produce EPS files
%


\maketitle
\tableofcontents
% BEGIN - DO NOT REMOVE THIS LINE



%%%%%%%%%%%%%%%%%%%%%%%
\section{Introduction}
%%%%%%%%%%%%%%%%%%%%%%%

This chapter is under development.

%%%%%%%%%%%%%%%%%%%%%%%%%%%%
\section{System requirement}
%%%%%%%%%%%%%%%%%%%%%%%%%%%%

Basically if you are installing \Rlogo{} from the
sources, you should be able to build a ACNUC socket server.
The socket server will build under a number of common Unix and Unix-alike
platforms. You will need several tools: programs are written in C thus 
you will need a
means of compiling C (as gcc compilation tools for linux or unix, Apple
Developer Tools  for MacOSX). You need as well library zlib and sockets
(standards on linux and unix).


%%%%%%%%%%%%%%%%%%%%%%%%%%%%%%%%%%%%%%%%%%%%%%%%%%%%%%%%%%%%%%
\section{Setting a local ACNUC database to be queried by the server}
%%%%%%%%%%%%%%%%%%%%%%%%%%%%%%%%%%%%%%%%%%%%%%%%%%%%%%%%%%%%%%

First of all yo need an ACNUC database, built by yourself or  downloaded from the PBIL ftp server.
An ACNUC database is composed of two sets of files:
\begin{enumerate}
	\item  the acnuc index files.
	\item  the database files (\textit{i.e.} flat files in EMBL/GenBank or SwissProt format).
\end{enumerate}

These two sets will be located  in the \texttt{index} and  \texttt{flat\_files} directories  respectively.

An example of an ACNUC database is available on the PBIL ftp server  at this url:
\url{ftp://pbil.univ-lyon1.fr/pub/seqinr/demoacnuc/acnucdatabase.tar.Z}.
 
You may install the database as it follows:
Let \texttt{ACNUC\_HOME} be the base directory for ACNUC installation.

\begin{Schunk}
\begin{Sinput}
 dir.create("./ACNUC_HOME", showWarning = FALSE)
\end{Sinput}
\end{Schunk}


Let ACNUC\_HOME/ACNUC\_DB be the  directory where you want to install the databases and ACNUC\_HOME/ACNUC\_DB/demoacnuc 
the directory where you want to install the demo database.

\begin{Schunk}
\begin{Sinput}
 dir.create("./ACNUC_HOME/ACNUC_DB", showWarning = FALSE)
 dir.create("./ACNUC_HOME/ACNUC_DB/demoacnuc", showWarning = FALSE)
\end{Sinput}
\end{Schunk}


\begin{itemize}
\item Dowload the ACNUC database in the ./ACNUC\_HOME/ACNUC\_DB/demoacnuc/ directory.

% Mettre eval=F apres premier download
\begin{Schunk}
\begin{Sinput}
 download.file("ftp://pbil.univ-lyon1.fr/pub/seqinr/demoacnuc/acnucdatabase.tar.Z", 
     destfile = "./ACNUC_HOME/ACNUC_DB/demoacnuc/acnucdatabase.tar.Z")
\end{Sinput}
\end{Schunk}


\item Uncompress and untar the \texttt{acnucdatabase.tar.Z} file 

\begin{Schunk}
\begin{Sinput}
 pwd <- getwd()
 setwd("./ACNUC_HOME/ACNUC_DB/demoacnuc/")
 system("gunzip -f acnucdatabase.tar.Z")
 system("tar -xvf acnucdatabase.tar")
 system("rm -f acnucdatabase.tar")
 setwd(pwd)
\end{Sinput}
\end{Schunk}

\end{itemize}
Now you sould get the following directories:
\begin{verbatim}
ACNUC_HOME/ACNUC_DB/demoacnuc/index
ACNUC_HOME/ACNUC_DB/demoacnuc/flat_files
\end{verbatim}

The directory \texttt{ACNUC\_HOME/ACNUC\_DB/demoacnuc} contains: 
\begin{Schunk}
\begin{Sinput}
 dir("./ACNUC_HOME/ACNUC_DB/demoacnuc")
\end{Sinput}
\begin{Soutput}
[1] "flat_files" "index"     
\end{Soutput}
\end{Schunk}

These directories contain respectively:

\begin{Schunk}
\begin{Sinput}
 dir("./ACNUC_HOME/ACNUC_DB/demoacnuc/index")
\end{Sinput}
\begin{Soutput}
 [1] "ACCESS"                  "AUTHOR"                 
 [3] "BIBLIO"                  "custom_qualifier_policy"
 [5] "EXTRACT"                 "HELP"                   
 [7] "HELP_WIN"                "KEYWORDS"               
 [9] "LOCUS"                   "LONGL"                  
[11] "MERES"                   "SHORTL"                 
[13] "SMJYT"                   "SPECIES"                
[15] "SUBSEQ"                  "TAXIDS"                 
[17] "TEXT"                   
\end{Soutput}
\begin{Sinput}
 dir("./ACNUC_HOME/ACNUC_DB/demoacnuc/flat_files")
\end{Sinput}
\begin{Soutput}
[1] "escherichia.dat" "id.log"          "yeast.dat"      
\end{Soutput}
\end{Schunk}


This database contains the complete genome of \textit{Escherichia coli} 
K12 W3110 and \textit{Saccharomyces cerevesiae}.

%%%%%%%%%%%%%%%%%%%%%%%%%%%%%%%%%%%%%%%%%%%%%%%%%%%%%%%%%%
\section{Build the ACNUC sockets server from the sources.}
%%%%%%%%%%%%%%%%%%%%%%%%%%%%%%%%%%%%%%%%%%%%%%%%%%%%%%%%%%

Once you have a local  ACNUC database available on your server you need to install the sockets server.

\subsection{Download the sources.}
%%%%%%%%%%%%%%%%%%%%%%%%%%%%%%%%%%

The code source of the racnucd server is available on the PBIL server  at this url:
\begin{verbatim}
http://pbil.univ-lyon1.fr/databases/acnuc/racnucd.html
\end{verbatim}
Alternatively you can download directly  the source from the ftp at:
\begin{verbatim}
ftp://pbil.univ-lyon1.fr/pub/acnuc/unix/racnucd.tar
\end{verbatim}

\subsection{Build the ACNUC sockets server.}
%%%%%%%%%%%%%%%%%%%%%%%%%%%%%%%%%%%%%%%%%%%%

You may install the racnucd server as it follows:
let ACNUC\_HOME/ACNUC\_SOFT/ be the base directory for the ACNUC softs.

\begin{Schunk}
\begin{Sinput}
 dir.create("./ACNUC_HOME/ACNUC_SOFT", showWarning = FALSE)
\end{Sinput}
\end{Schunk}

\begin{itemize}
\item Dowload the \texttt{racnucd.tar} file into ACNUC\_HOME/ACNUC\_SOFT.

% Mettre eval=F apres premier download
\begin{Schunk}
\begin{Sinput}
 download.file("ftp://pbil.univ-lyon1.fr/pub/acnuc/unix/racnucd.tar", 
     destfile = "./ACNUC_HOME/ACNUC_SOFT/racnucd.tar")
\end{Sinput}
\end{Schunk}

\item Untar the \texttt{racnucd.tar} file 

\begin{Schunk}
\begin{Sinput}
 setwd("./ACNUC_HOME/ACNUC_SOFT/")
 system("tar -xvf racnucd.tar")
 system("rm -f racnucd.tar")
 setwd(pwd)
\end{Sinput}
\end{Schunk}

\end{itemize}
Now you sould get the following directory:


\begin{Schunk}
\begin{Sinput}
 dir("./ACNUC_HOME/ACNUC_SOFT/")
\end{Sinput}
\begin{Soutput}
[1] "racnucd"
\end{Soutput}
\begin{Sinput}
 dir("./ACNUC_HOME/ACNUC_SOFT/racnucd/")
\end{Sinput}
\begin{Soutput}
 [1] "bit.c"                "dbplaces"             "dir_acnuc.h"         
 [4] "dir_io.c"             "dir_io.h"             "execute.c"           
 [7] "execute.h"            "extract.c"            "knowndbs"            
[10] "lngbit.c"             "makefile"             "md5.c"               
[13] "misc_acnuc.c"         "ordre.h"              "parser.c"            
[16] "prep_acnuc_requete.c" "pretty_seq.c"         "proc_requete.c"      
[19] "racnucd.ini"          "requete_acnuc.h"      "serveur.c"           
[22] "serveur.h"            "simext.h"             "use_acnuc.c"         
[25] "utilquery.c"          "zsockw.c"            
\end{Soutput}
\end{Schunk}

Go into \texttt{ACNUC\_HOME/ACNUC\_SOFT/racnucd/} and type \texttt{make}.
This should create the \textbf{racnucd} executable.

\begin{Schunk}
\begin{Sinput}
 setwd("./ACNUC_HOME/ACNUC_SOFT/racnucd")
 system("make")
 dir(pattern = "racnucd")
\end{Sinput}
\begin{Soutput}
[1] "racnucd"     "racnucd.ini"
\end{Soutput}
\begin{Sinput}
 setwd(pwd)
\end{Sinput}
\end{Schunk}

\subsection{Setting the ACNUC sockets server.}
%%%%%%%%%%%%%%%%%%%%%%%%%%%%%%%%%%%%%%%%%%%%%%

The server is configured by several parameters described in a configuration file \textbf{racnuc.ini}.
The \textbf{racnucd.ini} file is structued as follows:


\begin{Schunk}
\begin{Sinput}
 cat(readLines("./ACNUC_HOME/ACNUC_SOFT/racnucd/racnucd.ini"), 
     sep = "\n")
\end{Sinput}
\begin{Soutput}
port=5558
maxtime=8000
known_db_file=knowndbs
db_env_names=dbplaces
\end{Soutput}
\end{Schunk}

\begin{itemize}
\item \textbf{port} is the port of the socket server 
\item \textbf{maxtimle} is the time delay of the connection
\item \textbf{knowndbs} is a file containing the list of available databases
\item \textbf{dbplaces} is a file containing the path of the available databases
\end{itemize}



You may want to change the port of the socket server, according to the availabilities and restricttions on your machine.
For example , lets use the port 49152 in a new racnucd.new file.

\begin{Schunk}
\begin{Sinput}
 initline <- readLines("./ACNUC_HOME/ACNUC_SOFT/racnucd/racnucd.ini")
 initline[1] = "port=49152"
 writeLines(initline, "./ACNUC_HOME/ACNUC_SOFT/racnucd/racnucd.new")
 cat(readLines("./ACNUC_HOME/ACNUC_SOFT/racnucd/racnucd.new"), 
     sep = "\n")
\end{Sinput}
\begin{Soutput}
port=49152
maxtime=8000
known_db_file=knowndbs
db_env_names=dbplaces
\end{Soutput}
\end{Schunk}

 
\subsubsection{Configuring the knowndbs file.}
%%%%%%%%%%%%%%%%%%%%%%%%%%%%%%%%%%%%%%%%%

The \textbf{knowndbs} contains:

\begin{Schunk}
\begin{Sinput}
 cat(readLines("./ACNUC_HOME/ACNUC_SOFT/racnucd/knowndbs"), 
     sep = "\n")
\end{Sinput}
\begin{Soutput}
embl | on |    | EMBL sequence data library | 
swissprot   | on |  | UniProt |
\end{Soutput}
\end{Schunk}

Each line defines a database,  the four fields indicating respectively the name
 of the database, its status  (\textit{on} or \textit{off}), a tag and a short description.
 
You should set the files \textbf{knowndbs}  according to your installation.
Let's call the database you installed previously \textit{demoacnuc}.
Modify  the \textbf{knowndbs} as follows:
\begin{verbatim}
demoacnuc | on |    | Demo Database | 
\end{verbatim}

\begin{Schunk}
\begin{Sinput}
 writeLines("demoacnuc | on |    | Demo Database | ", "./ACNUC_HOME/ACNUC_SOFT/racnucd/knowndbs")
 knowndbs <- readLines("./ACNUC_HOME/ACNUC_SOFT/racnucd/knowndbs")
 cat(knowndbs, sep = "\n")
\end{Sinput}
\begin{Soutput}
demoacnuc | on |    | Demo Database | 
\end{Soutput}
\end{Schunk}

\subsubsection{Configuring the dbplaces file.}
%%%%%%%%%%%%%%%%%%%%%%%%%%%%%%%%%%%%%%%%%%%%%%


The \textbf{dbplaces} contains:

\begin{Schunk}
\begin{Sinput}
 dbplaces <- readLines("./ACNUC_HOME/ACNUC_SOFT/racnucd/dbplaces")
 cat(dbplaces, sep = "\n")
\end{Sinput}
\begin{Soutput}
#defines location of acnuc databases index files and flat files

setenv 	swissprot 	'/Users/mgouy/Documents/acnuc/petite/swissprot /Users/mgouy/Documents/acnuc/petite/swissprot'
setenv 	embl 	'/Users/mgouy/Documents/acnuc/petite/embl /Users/mgouy/Documents/acnuc/petite/embl'
\end{Soutput}
\end{Schunk}

Each line set the acnuc and gcgacnuc variables for each   database.


You should set the files \textbf{dbplaces}  according to your installation:
 modify  the \textbf{dbplaces} as follows:
\begin{verbatim}
setenv  demoacnuc       'ACNUC_HOME/ACNUC_DB/demoacnuc/index ACNUC_HOME/ACNUC_DB/demoacnuc/flat_files'
\end{verbatim}

\begin{Schunk}
\begin{Sinput}
 indexpath <- normalizePath("./ACNUC_HOME/ACNUC_DB/demoacnuc/index")
 ffpath <- normalizePath("./ACNUC_HOME/ACNUC_DB/demoacnuc/flat_files")
 newdb <- paste("setenv demoacnuc '", indexpath, " ", ffpath, 
     "'", sep = "", collapse = "")
 writeLines(newdb, "./ACNUC_HOME/ACNUC_SOFT/racnucd/dbplaces")
 dbplaces <- readLines("./ACNUC_HOME/ACNUC_SOFT/racnucd/dbplaces")
 cat(dbplaces, sep = "\n")
\end{Sinput}
\begin{Soutput}
setenv demoacnuc '/Users/lobry/seqinr/pkg/inst/doc/src/mainmatter/ACNUC_HOME/ACNUC_DB/demoacnuc/index /Users/lobry/seqinr/pkg/inst/doc/src/mainmatter/ACNUC_HOME/ACNUC_DB/demoacnuc/flat_files'
\end{Soutput}
\end{Schunk}

\subsubsection{Launch the server.}
%%%%%%%%%%%%%%%%%%%%%%%%%%%%%%%%%%


Finaly, in the ACNUC\_HOME/ACNUC\_SOFT/racnucd/ directory, lauche the server as follow :

\begin{Schunk}
\begin{Sinput}
 setwd("./ACNUC_HOME/ACNUC_SOFT/racnucd")
 system("./racnucd racnucd.new > racnucd.log &")
 Sys.sleep(1)
 system("ps | grep racnucd", intern = TRUE)
\end{Sinput}
\begin{Soutput}
[1] "14492  p2  S+     0:00.01 ./racnucd racnucd.new"  
[2] "14493  p2  S+     0:00.01 sh -c ps | grep racnucd"
[3] "14495  p2  S+     0:00.00 grep racnucd"           
\end{Soutput}
\begin{Sinput}
 cat(readLines("racnucd.log"), sep = "\n")
\end{Sinput}
\begin{Soutput}
*******************************************************
Start of remote acnuc server : Thu Nov  5 14:22:23 2009
\end{Soutput}
\begin{Sinput}
 setwd(pwd)
\end{Sinput}
\end{Schunk}

The server is now ready.

\subsection{Using seqinR to query your local socket server.}
%%%%%%%%%%%%%%%%%%%%%%%%%%%%%%%%%%%%%%%%%%%%%%%%%%%%%%%%%%%%

Launch \Rlogo{}, load the \texttt{seqinr} package  and type


\begin{verbatim}
choosebank(host="my_machine",port=49152,info=T)
\end{verbatim}

for example:


\begin{Schunk}
\begin{Sinput}
 library(seqinr)
 hostname <- "localhost"
 choosebank(host = hostname, port = 49152, info = TRUE)
\end{Sinput}
\begin{Soutput}
       bank status
1 demoacnuc     on
                                                                 info
1 ACNUC database example. (September 2007) Last Updated: Oct 15, 2007
\end{Soutput}
\end{Schunk}


You can query the database. For example:

\begin{Schunk}
\begin{Sinput}
 choosebank(bank = "demoacnuc", host = hostname, port = 49152)
 query("mylist", "k=rib@ prot@")
 mylist$nelem
\end{Sinput}
\begin{Soutput}
[1] 39
\end{Soutput}
\begin{Sinput}
 getName(mylist$req)
\end{Sinput}
\begin{Soutput}
 [1] "AP009048.PE25"   "AP009048.PE405"  "AP009048.PE830"  "AP009048.PE3223"
 [5] "AP009048.PE3465" "AP009048.PE3466" "AP009048.PE3516" "U00091.PE38"    
 [9] "U00093.PE119"    "U00093.PE123"    "U00094.PE65"     "U00094.PE87"    
[13] "U00094.PE262"    "U00094.PE393"    "U00094.PE400"    "X59720.PE36"    
[17] "Y13134.PE91"     "Y13134.PE272"    "Y13135.PE271"    "Y13137.PE286"   
[21] "Y13138.PE70"     "Y13138.PE198"    "Y13138.PE280"    "Y13139.PE53"    
[25] "Y13139.PE110"    "Y13139.PE316"    "Y13140.PE89"     "Z47047.PE177"   
[29] "Z47047.PE180"    "Z71256.PE178"    "Z71256.PE289"    "Z71256.PE313"   
[33] "Z71256.PE317"    "Z71256.PE534"    "Z71256.PE637"    "Z71256.PE694"   
[37] "Z71257.PE43"     "Z71257.PE75"     "Z71257.PE263"   
\end{Soutput}
\end{Schunk}


%%%%%%%%%%%%%%%%%%%%%%%%%%%%%%%%%%%%%%%%%%%%%%%%%%%%%%%%%%%%%
\section{Building your own  ACNUC database.}
%%%%%%%%%%%%%%%%%%%%%%%%%%%%%%%%%%%%%%%%%%%%%%%%%%%%%%%%%%%%%

One of the interest of a local server is to be able use your own ACNUC
database.


\subsection{Database flatfiles formats.}
%%%%%%%%%%%%%%%%%%%%%%%%%%%%%%%%%%%%%%%%

ACNUC database are build from flat files in several possible format : EMBL, Genbank or
SwissProt. Instructions to install ACNUC databases are given at this url :
\begin{verbatim}
http://pbil.univ-lyon1.fr/databases/acnuc/localinstall.html
\end{verbatim}


\subsection{Download the ACNUC dababase management tools.}
 
 
 
 
 The code source of the ACNUC tools server are available on the PBIL server  at this url:
\begin{verbatim}
ftp://pbil.univ-lyon1.fr/pub/acnuc/unix/acnucsoft.tar
\end{verbatim}

\subsection{Install the ACNUC dababase management tools.}
ANCUC management tools are described at this url :
\begin{verbatim}
http://pbil.univ-lyon1.fr/databases/acnuc/acnuc_gestion.html
\end{verbatim}

Let ACNUC\_HOME/ACNUC\_SOFT/tools be the base directory for the ACNUC tools.


\begin{Schunk}
\begin{Sinput}
 dir.create("./ACNUC_HOME/ACNUC_SOFT/tools", showWarning = FALSE)
\end{Sinput}
\end{Schunk}

\begin{itemize}
\item Dowload the \texttt{acnucsoft.tar} file into ACNUC\_HOME/ACNUC\_SOFT/tools.

% Mettre eval=F apres premier download
\begin{Schunk}
\begin{Sinput}
 download.file("ftp://pbil.univ-lyon1.fr/pub/acnuc/unix/acnucsoft.tar", 
     destfile = "./ACNUC_HOME/ACNUC_SOFT/tools/acnucsoft.tar")
\end{Sinput}
\end{Schunk}


\item Untar the \texttt{acnucsoft.tar} file 

\begin{Schunk}
\begin{Sinput}
 setwd("./ACNUC_HOME/ACNUC_SOFT/tools/")
 system("tar -xvf acnucsoft.tar")
 system("rm -f acnucsoft.tar")
 setwd(pwd)
\end{Sinput}
\end{Schunk}




\item Go into ACNUC\_SOFT/ and type;
\begin{verbatim}
make
\end{verbatim}
This should create the ACNUC management tools and ACNUC querying tools.

\begin{Schunk}
\begin{Sinput}
 setwd("./ACNUC_HOME/ACNUC_SOFT/tools/")
 system("make")
 dir()
\end{Sinput}
\begin{Soutput}
 [1] "acnuc2fasta.c"       "acnucf2c.c"          "acnucf2c.o"         
 [4] "acnucgener.c"        "arbrebin.c"          "arbrebin.o"         
 [7] "bit.c"               "bit.o"               "compressnewdiv.c"   
[10] "connectindex.c"      "conv_to_bigannots.c" "coperations.c"      
[13] "dir_acnuc.h"         "dir_io.c"            "dir_io.h"           
[16] "dir_io.o"            "dynlist.c"           "dynlist.o"          
[19] "extract.c"           "fortran_ex.f"        "gestion_acnuc.c"    
[22] "gestion_acnuc.o"     "hashacc.c"           "hashacc.o"          
[25] "initf.c"             "libcacnuc.a"         "lngbit.c"           
[28] "lngbit.o"            "makefile"            "mdshrt_lng.c"       
[31] "mdshrt_lng.o"        "misc_acnuc.c"        "misc_acnuc.o"       
[34] "ncbitaxo.h"          "newordalphab.c"      "pretty_seq.c"       
[37] "proc_requete.c"      "query.c"             "readncbitaxo.c"     
[40] "renamediv.c"         "simext.h"            "smjytload.c"        
[43] "sortsubseq.c"        "supold.c"            "testmatchindex.c"   
[46] "two_banks.c"         "two_banks.o"         "updatehelp.c"       
[49] "use_acnuc.c"         "use_acnuc.o"         "utilgener.c"        
[52] "utilgener.h"         "utilgener.o"         "utilgener2.c"       
[55] "utilgener2.o"        "utilquery.c"         "utilquery.o"        
[58] "voyage.c"           
\end{Soutput}
\begin{Sinput}
 setwd(pwd)
\end{Sinput}
\end{Schunk}

\end{itemize}
 
\subsection{Database building : index generation}
%%%%%%%%%%%%%%%%%%%%%%%%%%%%%%%%%%%%%%%%%%%%%%%%%


You can now build your own database.
All you need is a flat files
in EMBL, GenBank or SwissProt format. You can download a file example at :
\begin{verbatim}
ftp://pbil.univ-lyon1.fr/pub/seqinr/demoacnuc/escherichia_uniprot.dat.Z
\end{verbatim}

Let's use this SwissProt file to build your database
\begin{itemize}

\item Let ACNUC\_HOME/ACNUC\_DB/mydb be the directory for your databases.


\begin{Schunk}
\begin{Sinput}
 dir.create("./ACNUC_HOME/ACNUC_DB/mydb", showWarning = FALSE)
\end{Sinput}
\end{Schunk}
This directory should contain the  \textit{index} and \textit{flat\_files} directories.

\begin{Schunk}
\begin{Sinput}
 dir.create("./ACNUC_HOME/ACNUC_DB/mydb/index", showWarning = FALSE)
 dir.create("./ACNUC_HOME/ACNUC_DB/mydb/flat_files", showWarning = FALSE)
\end{Sinput}
\end{Schunk}

\item Download the \texttt{escherichia\_uniprot.dat.Z} file into ACNUC\_HOME/ACNUC\_DB/mydb/flat\_files.

% Mettre eval=F apres premier download
\begin{Schunk}
\begin{Sinput}
 download.file("ftp://pbil.univ-lyon1.fr/pub/seqinr/demoacnuc/escherichia_uniprot.dat.Z", 
     destfile = "./ACNUC_HOME/ACNUC_DB/mydb/flat_files/escherichia_uniprot.dat.Z")
\end{Sinput}
\end{Schunk}


\item Uncompress the \texttt{escherichia\_uniprot.dat.Z} file 

\begin{Schunk}
\begin{Sinput}
 setwd("./ACNUC_HOME/ACNUC_DB/mydb/flat_files/")
 system("gunzip -f escherichia_uniprot.dat.Z")
 setwd(pwd)
\end{Sinput}
\end{Schunk}


\item A simple building of the index can be done with the script  \texttt{buildindex.csh} available at:

\begin{verbatim}
ftp://pbil.univ-lyon1.fr/pub/seqinr/demoacnuc/buildindex.csh
\end{verbatim}


You can copy this file in ACNUC\_HOME/ACNUC\_DB/mydb and execute it by typing::

\begin{verbatim}
./buildindex.csh escherichia_uniprot
\end{verbatim}

% Mettre eval=F apres premier download
\begin{Schunk}
\begin{Sinput}
 download.file("ftp://pbil.univ-lyon1.fr/pub/seqinr/demoacnuc/buildindex.csh", 
     destfile = "./ACNUC_HOME/ACNUC_DB/mydb/buildindex.csh")
 setwd("./ACNUC_HOME/ACNUC_DB/mydb/")
 system("chmod +x ./buildindex.csh")
 system("./buildindex.csh escherichia_uniprot >  ./build.log")
 cat(readLines("build.log", 50), sep = "\n")
\end{Sinput}
\begin{Soutput}
 Build a protein database in:
 ============================
 ->/Users/lobry/seqinr/pkg/inst/doc/src/mainmatter/ACNUC_HOME/ACNUC_DB/mydb
 
ACNUC environment:
 ->/Users/lobry/seqinr/pkg/inst/doc/src/mainmatter/ACNUC_HOME/ACNUC_DB/mydb/index
 ->/Users/lobry/seqinr/pkg/inst/doc/src/mainmatter/ACNUC_HOME/ACNUC_DB/mydb/flat_files
 
ACNUC tools in:
 ->/Users/lobry/seqinr/pkg/inst/doc/src/mainmatter/ACNUC_HOME/ACNUC_DB/mydb/../../ACNUC_SOFT/tools
 
flat file: escherichia_uniprot.dat
 ->/Users/lobry/seqinr/pkg/inst/doc/src/mainmatter/ACNUC_HOME/ACNUC_DB/mydb/flat_files/escherichia_uniprot.dat
 
Begin to build index...
Thu Nov  5 14:22:43 CET 2009
=====================================
Initialise

=====================================
Generation des index

=====================================
run newordalphab


=====================================
run updatehelp
Thu Nov  5 14:22:43 CET 2009
Index have been sucessfully build.
=====================================
 
Testing the index:
==================
\end{Soutput}
\begin{Sinput}
 cat(tail(readLines("build.log"), 50), sep = "\n")
\end{Sinput}
\begin{Soutput}
 Build a protein database in:
 ============================
 ->/Users/lobry/seqinr/pkg/inst/doc/src/mainmatter/ACNUC_HOME/ACNUC_DB/mydb
 
ACNUC environment:
 ->/Users/lobry/seqinr/pkg/inst/doc/src/mainmatter/ACNUC_HOME/ACNUC_DB/mydb/index
 ->/Users/lobry/seqinr/pkg/inst/doc/src/mainmatter/ACNUC_HOME/ACNUC_DB/mydb/flat_files
 
ACNUC tools in:
 ->/Users/lobry/seqinr/pkg/inst/doc/src/mainmatter/ACNUC_HOME/ACNUC_DB/mydb/../../ACNUC_SOFT/tools
 
flat file: escherichia_uniprot.dat
 ->/Users/lobry/seqinr/pkg/inst/doc/src/mainmatter/ACNUC_HOME/ACNUC_DB/mydb/flat_files/escherichia_uniprot.dat
 
Begin to build index...
Thu Nov  5 14:22:43 CET 2009
=====================================
Initialise

=====================================
Generation des index

=====================================
run newordalphab


=====================================
run updatehelp
Thu Nov  5 14:22:43 CET 2009
Index have been sucessfully build.
=====================================
 
Testing the index:
==================
\end{Soutput}
\begin{Sinput}
 setwd(pwd)
\end{Sinput}
\end{Schunk}
You can check the building in the \texttt{build.log} file.

\end{itemize}




%%%%%%%%%%%%%%
\section{Misc} 
%%%%%%%%%%%%%

\subsection{Other tools for acnuc}

Several powerful tools dedicated to query ACNUC databases are available. 
The programs \textbf{query} and \textbf{query\_win} allow to query an ACNUC database according to the same
criteria than described  in \seqinr{}. It allows as well several functionality to extract biological data.
\textbf{query\_win} is a graphical version of query (\textit{cf} figure \ref{queryw}).
\textbf{query} is an command-line version which allows to query and ACNUC database through scripts.
Both \textbf{query} and  \textbf{query\_win} are available as a
\textit{client} or a \textit{local} application. More information on these programs can be found at:
\url{http://pbil.univ-lyon1.fr/software/query_win.html}

\begin{figure}
\fbox{
\begin{minipage}{\textwidth}
\includegraphics[width=\textwidth]{../figs/query_win_screenshot}
\caption{Screenshot of query\_win}
\label{queryw}
\end{minipage}
}
\end{figure}


Note:
The \textit{local} version of \textbf{query} is distributed with the ACNUC management  tools,
 thus it is already available in your ./ACNUC\_HOME/ACNUC\_SOFT/tools/ directory.
Before using it you need to set two environment  variables, \textit{acnuc} and \textit{gcganuc} :
\begin{verbatim}
setenv acnuc MYDATABASE/index
setenv gcgacnuc MYDATABASE/flat_files 
\end{verbatim}

where MYDATABASE is the path to the database you want to query ( for example: ./ACNUC\_HOME/ACNUC\_DB/demoacnuc/
 or ./ACNUC\_HOME/ACNUC\_DB/mydb/)
  
 




%%%%%%%%%%%%%%%%%%%%%%%%%%%%%%%%%%%%%%%%%%%%%%%%%%%%%
\section{Technical description of the racnucd daemon}
%%%%%%%%%%%%%%%%%%%%%%%%%%%%%%%%%%%%%%%%%%%%%%%%%%%%%
Technical information about the acnuc socket server is available at this url:
\url{http://pbil.univ-lyon1.fr/databases/acnuc/racnucd.html}.

%%%%%%%%%%%%%%%%%%%%%%%%%%%%%%%%%%%%%
\section{ACNUC remote access protocol}
%%%%%%%%%%%%%%%%%%%%%%%%%%%%%%%%%%%%%
Description of the socket communication protocol with acnuc is availble at this url:
\url{http://pbil.univ-lyon1.fr/databases/acnuc/remote_acnuc.html}




\section{Citation} 
%%%%%%%%%%%%%

You can use a citation along these lines:

\vspace{0.2cm}

\noindent\fbox{\begin{minipage}{\textwidth}
\noindent Sequences from [\textit{cite your source of data}] were structured under the 
ACNUC model \cite{acnuc1985}, hosted [at \textit{give your URL if
public}] by an ACNUC server \cite{acnuc2007} and analyzed with 
the \seqinr{} client \cite{seqinr} under the \Rlogo{} statistical environment \cite{RfromR}.
\end{minipage}}

\vspace{0.2cm}

For \LaTeX{} users, these references are available in the \texttt{book.bib} file that ships
with \seqinr{} in the \texttt{seqinr/doc/src/config/} folder. To locate this file on your
computer try:

\begin{Schunk}
\begin{Sinput}
 (seqinrloc <- normalizePath(.path.package("seqinr")))
\end{Sinput}
\begin{Soutput}
[1] "/Users/lobry/seqinr/pkg.Rcheck/seqinr"
\end{Soutput}
\begin{Sinput}
 setwd(seqinrloc)
 dir()
\end{Sinput}
\begin{Soutput}
 [1] "abif"        "CITATION"    "data"        "DESCRIPTION" "doc"        
 [6] "help"        "html"        "INDEX"       "libs"        "Meta"       
[11] "R"           "sequences"  
\end{Soutput}
\begin{Sinput}
 setwd("./doc/src/config")
 dir()
\end{Sinput}
\begin{Soutput}
[1] "atxy.sty"        "authors.tex"     "book.bib"        "commonrnw.rnw"  
[5] "commontex.tex"   "sessionInfo.rnw"
\end{Soutput}
\begin{Sinput}
 cat(readLines("book.bib", n = 5), sep = "\n")
\end{Sinput}
\begin{Soutput}
@incollection{seqinr,
    author = {Charif, D. and Lobry, J.R.},
    title = {Seqin{R} 1.0-2: a contributed package to the {R} project for statistical computing devoted to biological sequences retrieval and analysis.},
    booktitle = {Structural approaches to sequence evolution: Molecules, networks, populations},
    year = {2007},
\end{Soutput}
\end{Schunk}

%
% For Sweave users, go back home:
%


\section*{Session Informations}

\begin{scriptsize}

This part was compiled under the following \Rlogo{}~environment:

\begin{itemize}\raggedright
  \item R version 2.10.0 (2009-10-26), \verb|i386-apple-darwin8.11.1|
  \item Locale: \verb|fr_FR.UTF-8/fr_FR.UTF-8/fr_FR.UTF-8/C/C/C|
  \item Base packages: base, datasets, graphics, grDevices, grid,
    methods, stats, utils
  \item Other packages: ade4~1.4-13, ape~2.4, grImport~0.4-4,
    MASS~7.3-3, quadprog~1.4-11, seqinr~2.0-7, tseries~0.10-21,
    XML~2.6-0, xtable~1.5-5, zoo~1.5-8
  \item Loaded via a namespace (and not attached): gee~4.13-14,
    lattice~0.17-26, nlme~3.1-96, tools~2.10.0
\end{itemize}
There were two compilation steps:

\begin{itemize}
  \item \Rlogo{} compilation time was: Thu Nov  5 14:22:43 2009
  \item \LaTeX{} compilation time was: \today
\end{itemize}

\end{scriptsize}

% END - DO NOT REMOVE THIS LINE

%%%%%%%%%%%%  BIBLIOGRAPHY %%%%%%%%%%%%%%%%%
\clearpage
\addcontentsline{toc}{section}{References}
\bibliographystyle{plain}
\bibliography{../config/book}
\end{document}
