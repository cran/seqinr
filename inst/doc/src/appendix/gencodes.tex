\documentclass{article}
\usepackage{graphicx} % Extended graphics inclusions
\usepackage{float}
\usepackage{url} % For \url{}
\usepackage{../config/atxy} % For front cover
\usepackage{amsfonts} % Needed for some fonts
\usepackage[usenames]{color} % Needed for colored R input/output
\usepackage{pdfcolmk} % Correct some problems with the color stack


\title{Genetic codes}
\author{Lobry, J.R.}

\usepackage{/Library/Frameworks/R.framework/Resources/share/texmf/Sweave}
\begin{document}
%
% To change the R input/output style:
%
\definecolor{Soutput}{rgb}{0,0,0.56}
\definecolor{Sinput}{rgb}{0.56,0,0}
\DefineVerbatimEnvironment{Sinput}{Verbatim}
{formatcom={\color{Sinput}},fontsize=\footnotesize, baselinestretch=0.75}
\DefineVerbatimEnvironment{Soutput}{Verbatim}
{formatcom={\color{Soutput}},fontsize=\footnotesize, baselinestretch=0.75}
%
% This removes the extra spacing after code and output chunks in Sweave,
% but keeps the spacing around the whole block.
%
\fvset{listparameters={\setlength{\topsep}{0pt}}}
\renewenvironment{Schunk}{\vspace{\topsep}}{\vspace{\topsep}}
%
% Rlogo
%
\newcommand{\Rlogo}{\protect\includegraphics[height=1.8ex,keepaspectratio]{../figs/Rlogo.pdf}}
%
% Shortcut for seqinR:
%
\newcommand{\seqinr}{\texttt{seqin\bf{R}}}
\newcommand{\Seqinr}{\texttt{Seqin\bf{R}}}
\fvset{fontsize= \scriptsize}
%
% R output options and libraries to be loaded.
%
%
%  Sweave Options
%
% Put all figures in the fig folder and start the name with current file name.
% Do not produce EPS files
%


\maketitle
% BEGIN - DO NOT REMOVE THIS LINE

\section{Standard genetic code}

The standard genetic code given in table \ref{stdcode} was produced with the
following \Rlogo{}~code and inserted with \verb|\input{../tables/stdcode.tex}|
within this \LaTeX~document and referenced as \verb|\ref{stdcode}| in
the text.

\begin{Schunk}
\begin{Sinput}
 tablecode(latexfile = "../tables/stdcode.tex", label = "stdcode", 
     size = "small")
\end{Sinput}
\end{Schunk}

\input{../tables/stdcode.tex}

\section{Available genetic code numbers}

The genetic code numbers are those from the NCBI\footnote{
National Center for Biotechnology Information, Bethesda, Maryland, U.S.A.
} (\url{http://130.14.29.110/Taxonomy/Utils/wprintgc.cgi?mode=c}).
This compilation from Andrzej (Anjay) Elzanowski, Jim Ostell, Detlef Leipe, 
and Vladimir Soussov is based primarily on two previous reviews
\cite{OsawaS1992, JukesTH1993}.

\begin{Schunk}
\begin{Sinput}
 codes <- SEQINR.UTIL$CODES.NCBI
 availablecodes <- which(codes$CODES != "deleted")
 codes[availablecodes, "ORGANISMES", drop = FALSE]
\end{Sinput}
\begin{Soutput}
                           ORGANISMES
1                            standard
2            vertebrate.mitochondrial
3                 yeast.mitochondrial
4  protozoan.mitochondrial+mycoplasma
5          invertebrate.mitochondrial
6               ciliate+dasycladacean
9   echinoderm+flatworm.mitochondrial
10                           euplotid
11             bacterial+plantplastid
12                   alternativeyeast
13             ascidian.mitochondrial
14  alternativeflatworm.mitochondrial
15                         blepharism
16        chlorophycean.mitochondrial
21            trematode.mitochondrial
22          scenedesmus.mitochondrial
23       hraustochytrium.mitochondria
\end{Soutput}
\end{Schunk}

The tables of variant genetic codes outlining the differences were produced with the
following \Rlogo{}~code:

\begin{Schunk}
\begin{Sinput}
 cdorder <- paste(paste(rep(s2c("tcag"), each = 16), s2c("tcag"), 
     sep = ""), rep(s2c("tcag"), each = 4), sep = "")
 stdcode <- sapply(lapply(cdorder, s2c), translate, numcode = 1)
 for (cd in availablecodes[-1]) {
     Tfile <- paste("../tables/codnum", cd, ".tex", sep = "")
     preemph <- "\\textcolor{red}{\\textbf{"
     postemph <- "}}"
     stcodon <- (stdcode == sapply(lapply(cdorder, s2c), translate, 
         numcode = cd))
     pre <- ifelse(stcodon, "", preemph)
     post <- ifelse(stcodon, "", postemph)
     tablecode(numcode = cd, latexfile = Tfile, size = "small", 
         preaa = pre, postaa = post)
     cat(paste("\\input{", Tfile, "}", sep = ""), sep = "\n")
 }
\end{Sinput}
\input{../tables/codnum2.tex}
\input{../tables/codnum3.tex}
\input{../tables/codnum4.tex}
\input{../tables/codnum5.tex}
\input{../tables/codnum6.tex}
\input{../tables/codnum9.tex}
\input{../tables/codnum10.tex}
\input{../tables/codnum11.tex}
\input{../tables/codnum12.tex}
\input{../tables/codnum13.tex}
\input{../tables/codnum14.tex}
\input{../tables/codnum15.tex}
\input{../tables/codnum16.tex}
\input{../tables/codnum21.tex}
\input{../tables/codnum22.tex}
\input{../tables/codnum23.tex}\end{Schunk}


\section*{Session Informations}

This part was compiled under the following \Rlogo{}~environment:

\begin{itemize}
  \item R version 2.8.0 Under development (unstable) (2008-05-11 r45672), \verb|i386-apple-darwin8.8.2|
  \item Locale: \verb|C|
  \item Base packages: base, datasets, grDevices, graphics, methods,
    stats, utils
  \item Other packages: MASS~7.2-42, ade4~1.4-7, ape~2.2,
    nlme~3.1-88, seqinr~1.1-6, xtable~1.5-2
  \item Loaded via a namespace (and not attached): grid~2.8.0,
    lattice~0.17-7
\end{itemize}
There were two compilation steps:

\begin{itemize}
  \item \Rlogo{} compilation time was: Mon May 12 16:00:49 2008
  \item \LaTeX{} compilation time was: \today
\end{itemize}

%
% END - DO NOT REMOVE THIS LINE
%
%%%%%%%%%%%%  BIBLIOGRAPHY %%%%%%%%%%%%%%%%%
\clearpage
\addcontentsline{toc}{section}{References}
\bibliographystyle{plain}
\bibliography{../config/book}
\end{document}
