\documentclass{article}
\usepackage{graphicx} % Extended graphics inclusions
\usepackage{float}
\usepackage{url} % For \url{}
\usepackage{../config/atxy} % For front cover
\usepackage{amsfonts} % Needed for some fonts
\usepackage[usenames]{color} % Needed for colored R input/output
\usepackage{pdfcolmk} % Correct some problems with the color stack


\title{Informations about databases available at pbil}
\author{Lobry, J.R.}

\usepackage{/Library/Frameworks/R.framework/Resources/share/texmf/Sweave}
\begin{document}
%
% To change the R input/output style:
%
\definecolor{Soutput}{rgb}{0,0,0.56}
\definecolor{Sinput}{rgb}{0.56,0,0}
\DefineVerbatimEnvironment{Sinput}{Verbatim}
{formatcom={\color{Sinput}},fontsize=\footnotesize, baselinestretch=0.75}
\DefineVerbatimEnvironment{Soutput}{Verbatim}
{formatcom={\color{Soutput}},fontsize=\footnotesize, baselinestretch=0.75}
%
% Rlogo
%
\newcommand{\Rlogo}{\protect\includegraphics[height=1.8ex,keepaspectratio]{../figs/Rlogo.pdf}}
%
% Shortcut for seqinR:
%
\newcommand{\seqinr}{\texttt{seqin\bf{R}}}
\newcommand{\Seqinr}{\texttt{Seqin\bf{R}}}
\fvset{fontsize= \scriptsize}
%
% R output options and libraries to be loaded.
%
%
%  Sweave Options
%
% Put all figures in the fig folder and start the name with current file name.
% Do not produce EPS files
%


\maketitle
% BEGIN - DO NOT REMOVE THIS LINE
\label{subsequence}

\section{Introduction}
This section was compiled on \today. The list of available database at pbil 
(\url{http://pbil.univ-lyon1.fr/}) was:

\begin{Schunk}
\begin{Sinput}
 banknames <- choosebank()
 banknames
\end{Sinput}
\begin{Soutput}
 [1] "genbank"      "embl"         "emblwgs"      "swissprot"   
 [5] "ensembl"      "refseq"       "nrsub"        "hobacnucl"   
 [9] "hobacprot"    "hovernucl"    "hoverprot"    "hogennucl"   
[13] "hogenprot"    "hoverclnu"    "hoverclpr"    "homolensprot"
[17] "homolensnucl" "greview"      "emglib"       "HAMAPnucl"   
[21] "HAMAPprot"    "hoppsigen"    "nurebnucl"    "nurebprot"   
[25] "taxobacgen"  
\end{Soutput}
\end{Schunk}

This \LaTeX~file was automatically generated by the following \Rlogo{}~code:

\begin{Schunk}
\begin{Sinput}
 for (b in banknames) {
     cat(paste("\\section{", b, "}"), sep = "\n")
     choosebank(b)
     bankdetails <- sapply(banknameSocket$details, stresc, 
         USE.NAMES = FALSE)
     cat("\\subsection{Bank details}", sep = "\n")
     cat(bankdetails, sep = "\\\\\n")
     cat("\n")
     cat("\\subsection{Type names}", sep = "\n")
     types <- getType()
     if (is.null(nrow(types))) {
         cat("There are no subsequence type in this database", 
             sep = "\n")
     }
     else {
         cat("\\noindent\\begin{tabular}{ll}", sep = "\n")
         cat("\\hline \\hline", sep = "\n")
         cat("name & description\\\\", sep = "\n")
         cat("\\hline", sep = "\n")
         for (i in 1:nrow(types)) cat(paste(stresc(types[i, 
             "sname"]), "&", stresc(types[i, "libel"]), "\\\\"), 
             sep = "\n")
         cat("\\hline \\hline", sep = "\n")
         cat("\\end{tabular}", sep = "\n")
         cat("\n")
     }
     closebank()
 }
\end{Sinput}
\section{ genbank }
\subsection{Bank details}
GenBank Rel. 160 (15 June 2007) Last Updated: Jul  7, 2007\\
77,927,415,059 bases; 73,849,130 sequences; 4,241,874 subseqs; 470,139 refers.\\
Software by M. Gouy, Lab. Biometrie et Biologie Evolutive, Universite Lyon I

\subsection{Type names}
\noindent\begin{tabular}{ll}
\hline \hline
name & description\\
\hline
CDS & .PE protein coding region \\
LOCUS & sequenced DNA fragment \\
MISC\_RNA & .RN other structural RNA coding region \\
RRNA & .RR mature ribosomal RNA \\
SCRNA & .SC small cytoplasmic RNA \\
SNRNA & .SN small nuclear RNA \\
TRNA & .TR mature transfer RNA \\
\hline \hline
\end{tabular}

\section{ embl }
\subsection{Bank details}
EMBL Library Release 91 (June 2007) Last Updated: Jul  7, 2007\\
78,294,203,189 bases; 73,908,519 sequences; 4,244,486 subseqs; 466,401 refers.\\
Software by M. Gouy, Laboratoire de biometrie, Universite Lyon I

\subsection{Type names}
\noindent\begin{tabular}{ll}
\hline \hline
name & description\\
\hline
CDS & .PE protein coding region \\
ID & Locus entry \\
MISC\_RNA & .RN other structural RNA coding region \\
RRNA & .RR Ribosomal RNA coding gene \\
SCRNA & .SC small cytoplasmic RNA \\
SNRNA & .SN small nuclear RNA \\
TRNA & .TR Transfer RNA coding gene \\
\hline \hline
\end{tabular}

\section{ emblwgs }
\subsection{Bank details}
EMBL Whole Genome Shotgun sequences Release 91  (June 2007)\\
93,563,351,281 bases; 23,401,573 sequences; 953,944 subseqs; 322 refers.\\
Retrieval software by M. Gouy, Biometrie et Biologie Evolutive, Univ Lyon I.

\subsection{Type names}
\noindent\begin{tabular}{ll}
\hline \hline
name & description\\
\hline
CDS & .PE protein coding region \\
ID & EMBL sequence data library entry \\
MISC\_RNA & .RN other structural RNA coding region \\
RRNA & .RR ribosomal RNA coding region \\
SCRNA & .SC small cytoplasmic RNA coding region \\
SNRNA & .SN small nuclear RNA coding region \\
TRNA & .TR transfer RNA coding region \\
\hline \hline
\end{tabular}

\section{ swissprot }
\subsection{Bank details}
UniProt Rel. 11 (SWISS-PROT 53 + TrEMBL 36): Last Updated: Jul  6, 2007\\
1,550,505,424 amino acids; 4,736,514 sequences; 274,467 references.\\
Non-redundant compilation of SWISS-PROT + TrEMBL\\
Software by M. Gouy \& L. Duret, Laboratoire de biometrie, Universite Lyon I.

\subsection{Type names}
There are no subsequence type in this database
\section{ ensembl }
\subsection{Bank details}
Ensembl databases release 41\\
Espece                                  Release/\#CDS(1)/STOP(2)/N(3)/miss(4)\\
Aedes aegypti                           41\_1a 11360/0/2/0 (0\%/0\%/0\%)\\
Anopheles gambiae                       41\_3d 13510/0/31/0 (0\%/0\%/0\%)\\
Apis mellifera                          38\_2d 27755/1/269/0 (0\%/0\%/0\%)\\
Bos taurus                              41\_2 32556/7/620/12 (0\%/1\%/0\%)\\
Caenorhabditis elegans                  41\_160 25218/1/0/0 (0\%/0\%/0\%)\\
Canis familiaris                        41\_1j 29813/0/0/0 (0\%/0\%/0\%)\\
Caenorhabditis briggsae                 25 14712/0/23/1 (0\%/0\%/0\%)\\
Ciona intestinalis                      41\_2c 20000/0/128/0 (0\%/0\%/0\%)\\
Ciona savignyi                          41\_2b 20150/1/27/0 (0\%/0\%/0\%)\\
Danio rerio                             41\_6b 36065/5/361/0 (0\%/1\%/0\%)\\
Dasypus novemcinctus                    40\_1 13567/12/8857/0 (0\%/65\%/0\%)\\
Drosophila melanogaster                 41\_43 19577/33/1/0 (0\%/0\%/0\%)\\
Echinops telfairi                       40\_1 14309/8/9348/0 (0\%/65\%/0\%)\\
Gallus gallus                           41\_1p 20667/13/455/0 (0\%/2\%/0\%)\\
Gasterosteus aculeatus                  41\_1a 27181/13/138/0 (0\%/0\%/0\%)\\
Homo sapiens                            41\_36c 47004/41/6/0 (0\%/0\%/0\%)\\
Loxodonta africana                      40\_1 14366/10/9618/0 (0\%/66\%/0\%)\\
Macaca mulatta                          41\_10a 36446/14/491/0 (0\%/1\%/0\%)\\
Monodelphis domestica                   41\_3a 30358/0/80/0 (0\%/0\%/0\%)\\
Mus musculus                            41\_36b 29026/34/2/0 (0\%/0\%/0\%)\\
Oryctolagus cuniculus                   41\_1a 13705/4/8615/0 (0\%/62\%/0\%)\\
Oryzias latipes                         41\_1 25880/0/546/0 (0\%/2\%/0\%)\\
Pan troglodytes                         41\_21 32667/4/739/0 (0\%/2\%/0\%)\\
Rattus norvegicus                       41\_34k 32996/34/686/0 (0\%/2\%/0\%)\\
Saccharomyces cerevisiae                41\_1d 4767/2/0/0 (0\%/0\%/0\%)

\subsection{Type names}
\noindent\begin{tabular}{ll}
\hline \hline
name & description\\
\hline
3'INT & .3I 3'intron \\
3'NCR & .3F  3'-non coding region \\
5'INT & .5I 5'intron \\
5'NCR & .5F  5'-non coding region \\
CDS & .PE protein coding region \\
ID & EMBL sequence data library entry \\
INT\_INT & .IN  internal intron \\
MISC\_RNA & .RN other structural RNA coding region \\
MRNA & .RN mRNA \\
RRNA & .RR ribosomal RNA coding region \\
SCRNA & .SC small cytoplasmic RNA coding region \\
SNRNA & .SN small nuclear RNA coding region \\
TRNA & .TR transfer RNA coding region \\
\hline \hline
\end{tabular}

\section{ refseq }
\subsection{Bank details}
RefSeq 15.0 (1 January 2006) Last Updated: Jan 23, 2006\\
1,055,245,496 bases; 625,928 sequences; 254,162 subseqs; 205,831 refers.\\
Software by M. Gouy \& M. Jacobzone, Laboratoire de biometrie, Universite Lyon I

\subsection{Type names}
\noindent\begin{tabular}{ll}
\hline \hline
name & description\\
\hline
3'INT & .3I 3'intron \\
3'NCR & .3F  3'-non coding region \\
5'INT & .5I 5'intron \\
5'NCR & .5F  5'-non coding region \\
CDS & .PE protein coding region \\
INT\_INT & .IN  internal intron \\
LOCUS & sequenced DNA fragment \\
MISC\_RNA & .RN other structural RNA coding region \\
RRNA & .RR ribosomal RNA coding region \\
SCRNA & .SC small cytoplasmic RNA coding region \\
SNRNA & .SN small nuclear RNA coding region \\
TRNA & .TR transfer RNA coding region \\
\hline \hline
\end{tabular}

\section{ nrsub }
\subsection{Bank details}
NRSub database release 10.1 (December 1997)\\
Bacillus subtilis complete genome\\
Sequence data taken from the SubtiList database\\
Institut Pasteur - Unite de Regulation de l'Expression Genetique\\
Extra annotations provided by G. Perriere\\
Laboratoire BGBP - Universite Claude Bernard, Lyon 1

\subsection{Type names}
\noindent\begin{tabular}{ll}
\hline \hline
name & description\\
\hline
CDS & .PE protein coding region \\
ID & EMBL sequence data library entry \\
MISC\_RNA & .RN other structural RNA coding region \\
RRNA & .RR ribosomal RNA coding region \\
SCRNA & .SC small cytoplasmic RNA coding region \\
SNRNA & .SN small nuclear RNA coding region \\
TRNA & .TR transfer RNA coding region \\
\hline \hline
\end{tabular}

\section{ hobacnucl }
\subsection{Bank details}
HOBACGEN - genomic data - Release 10 (February 12 2002)\\
432,023,804 bases; 168,814 sequences; 293,669 subseqs; 52,735 references.\\
\\
Bacteria + Archaea + Saccharomyces cerevisiae\\
Genomic data from EMBL Release 69 (December 2001)\\


\subsection{Type names}
\noindent\begin{tabular}{ll}
\hline \hline
name & description\\
\hline
CDS & .PE protein coding region \\
ID & EMBL sequence data library entry \\
MISC\_RNA & .RN other structural RNA coding region \\
RRNA & .RR ribosomal RNA coding region \\
SCRNA & .SC small cytoplasmic RNA coding region \\
SNRNA & .SN small nuclear RNA coding region \\
TRNA & .TR transfer RNA coding region \\
\hline \hline
\end{tabular}

\section{ hobacprot }
\subsection{Bank details}
HOBACGEN - protein data - Release 10 (February 12 2002)\\
79,755,852 amino acids; 260,025 sequences; 37,383 references.\\
\\
Bacteria + Archaea + Saccharomyces cerevisiae\\
Protein data from SWISS-PROT 40 + TrEMBL 19 + TrEMBL\_NEW: January 25, 2002\\
\\
Software: M. Gouy \& M. Jacobzone\\
Data maintenance: L. Duret \& G. Perriere\\
\\
Laboratoire de Biometrie et Biologie Evolutive\\
UMR CNRS 5558, Universite Claude Bernard - Lyon 1\\
43, bd du 11 Novembre 1918 F-69622 Villeurbanne Cedex\\


\subsection{Type names}
There are no subsequence type in this database
\section{ hovernucl }
\subsection{Bank details}
HOVERGEN - genomic data - Release 48 (May 24 2007) Last Updated: May 24, 2007\\
2,500,248,516 bases; 541,405 sequences; 1,005,089 subseqs; 117,556 refers.\\
\\
Vertebrate (chordata)\\
Genomic data from EMBL Library Release 90 (March 2007)\\
\\
Retrieval software by M. Gouy \& M. Jacobzone, Lab. de Biometrie, UCB Lyon.\\
Data maintenance: L. Duret \& S. Penel\\


\subsection{Type names}
\noindent\begin{tabular}{ll}
\hline \hline
name & description\\
\hline
3'INT & .3I 3'intron \\
3'NCR & .3F  3'-non coding region \\
5'INT & .5I 5'intron \\
5'NCR & .5F  5'-non coding region \\
CDS & .PE protein coding region \\
ID & EMBL sequence data library entry \\
INT\_INT & .IN  internal intron \\
MISC\_RNA & .RN other structural RNA coding region \\
RRNA & .RR ribosomal RNA coding region \\
SCRNA & .SC small cytoplasmic RNA coding region \\
SNRNA & .SN small nuclear RNA coding region \\
TRNA & .TR transfer RNA coding region \\
\hline \hline
\end{tabular}

\section{ hoverprot }
\subsection{Bank details}
HOVERGEN - protein data - Release 48 (May 24 2007) Last Updated: May 24, 2007\\
142,891,140 amino acids; 415,383 sequences; 114,560 references.\\
\\
Vertebrate (chordata)	\\
Protein data from UniProt Rel. 10 (SWISS-PROT 52 + TrEMBL 35) May 2007\\
\\
Software: M. Gouy \& M. Jacobzone\\
Data maintenance: L. Duret \& S. Penel\\
\\
Laboratoire de Biometrie et Biologie Evolutive\\
UMR CNRS 5558, Universite Claude Bernard - Lyon 1\\
43, bd du 11 Novembre 1918 F-69622 Villeurbanne Cedex\\


\subsection{Type names}
There are no subsequence type in this database
\section{ hogennucl }
\subsection{Bank details}
HOGENOM - genomic data - Release 03 (Oct 14 2005) Last Updated: Nov  7, 2005\\
2,538,433,251 bases; 227,950 sequences; 4,136,134 subseqs; 82,281 refers.\\
\\
Fully Sequenced Organisms\\
Protein data from http://www.ebi.ac.uk/proteome/ (August, 2005)\\
Genomic data from GenomeReview  (June 2005)\\
and  EMBL (June 2005)\\
( 263 fully sequenced organisms)\\
\\
Retrieval software by M. Gouy \& M. Jacobzone, Lab. de Biometrie, UCB Lyon.\\
Data maintenance: L. Duret \& S. Penel\\
\\
Laboratoire de Biometrie et Biologie Evolutive\\
UMR CNRS 5558, Universite Claude Bernard - Lyon 1\\
43, bd du 11 Novembre 1918 F-69622 Villeurbanne Cedex\\


\subsection{Type names}
\noindent\begin{tabular}{ll}
\hline \hline
name & description\\
\hline
ID & EMBL sequence data library entry \\
CDS & .PE protein coding region \\
TRNA & .TR transfer RNA coding region \\
RRNA & .RR ribosomal RNA coding region \\
MISC\_RNA & .RN other structural RNA coding region \\
SCRNA & .SC small cytoplasmic RNA coding region \\
SNRNA & .SN small nuclear RNA coding region \\
3'INT & .3I 3'intron \\
3'NCR & .3F  3'-non coding region \\
5'INT & .5I 5'intron \\
5'NCR & .5F  5'-non coding region \\
INT\_INT & .IN  internal intron \\
\hline \hline
\end{tabular}

\section{ hogenprot }
\subsection{Bank details}
HOGENOM - protein data - Release 03 (Oct 14 2005) Last Updated: Mar 10, 2006\\
339,891,443 amino acids; 950,216 sequences; 92,805 references.\\
\\
Fully Sequenced Organisms\\
Protein data from http://www.ebi.ac.uk/proteome/ (August 2005)\\
( 263 fully sequenced organisms)\\
\\
Retrieval software by M. Gouy \& M. Jacobzone, Lab. de Biometrie, UCB Lyon.\\
Data maintenance: L. Duret \& S. Penel\\
\\
Laboratoire de Biometrie et Biologie Evolutive\\
UMR CNRS 5558, Universite Claude Bernard - Lyon 1\\
43, bd du 11 Novembre 1918 F-69622 Villeurbanne Cedex\\


\subsection{Type names}
There are no subsequence type in this database
\section{ hoverclnu }
\subsection{Bank details}
HOVERGEN CLEAN - genomic data - Release 46 (Jun 10 2004)\\
894,369,756 bases; 312,987 sequences; 796,415 subseqs; 99,342 refers.\\
\\
Vertebrate (chordata)\\
Genomic data from EMBL Release 78  (March 2004)\\
\\
Retrieval software by M. Gouy \& M. Jacobzone, Lab. de Biometrie, UCB Lyon.\\
Data maintenance: L. Duret \& S. Penel\\


\subsection{Type names}
\noindent\begin{tabular}{ll}
\hline \hline
name & description\\
\hline
3'INT & .3I 3'intron \\
3'NCR & .3F  3'-non coding region \\
5'INT & .5I 5'intron \\
5'NCR & .5F  5'-non coding region \\
CDS & .PE protein coding region \\
ID & EMBL sequence data library entry \\
INT\_INT & .IN  internal intron \\
MISC\_RNA & .RN other structural RNA coding region \\
RRNA & .RR ribosomal RNA coding region \\
SCRNA & .SC small cytoplasmic RNA coding region \\
SNRNA & .SN small nuclear RNA coding region \\
TRNA & .TR transfer RNA coding region \\
\hline \hline
\end{tabular}

\section{ hoverclpr }
\subsection{Bank details}
HOVERGEN CLEAN - protein data - Release 46 (Jun 10 2004)\\
75,885,664 amino acids; 219,552 sequences; 89,885 references.\\
\\
Vertebrate (chordata)	\\
Protein data from SWISS-PROT Rel. 43  + TrEMBL Rel. 26 + TrEMBL\_NEW: May 17, 2004\\
\\
Software: M. Gouy \& M. Jacobzone\\
Data maintenance: L. Duret \& S. Penel\\
\\
Laboratoire de Biometrie et Biologie Evolutive\\
UMR CNRS 5558, Universite Claude Bernard - Lyon 1\\
43, bd du 11 Novembre 1918 F-69622 Villeurbanne Cedex\\


\subsection{Type names}
There are no subsequence type in this database
\section{ homolensprot }
\subsection{Bank details}
HOMOLENS 3 - Homologous genes from Ensembl Last Updated: Jan 19, 2007\\
224,528,520 amino acids; 474,339 sequences; 0 references.\\
\\
Ensembl 41 Organisms Translated CDS\\
Aedes aegypti                           41\_1a 11360/0/2/0 (0\%/0\%/0\%)\\
Anopheles gambiae                       41\_3d 13510/0/31/0 (0\%/0\%/0\%)\\
Apis mellifera                          38\_2d 27755/1/269/0 (0\%/0\%/0\%)\\
Bos taurus                              41\_2 32556/7/620/12 (0\%/1\%/0\%)\\
Caenorhabditis elegans                  41\_160 25218/1/0/0 (0\%/0\%/0\%)\\
Canis familiaris                        41\_1j 29813/0/0/0 (0\%/0\%/0\%)\\
Caenorhabditis briggsae                 25 14712/0/23/1 (0\%/0\%/0\%)\\
Ciona intestinalis                      41\_2c 20000/0/128/0 (0\%/0\%/0\%)\\
Ciona savignyi                          41\_2b 20150/1/27/0 (0\%/0\%/0\%)\\
Danio rerio                             41\_6b 36065/5/361/0 (0\%/1\%/0\%)\\
Dasypus novemcinctus                    40\_1 13567/12/8857/0 (0\%/65\%/0\%)\\
Drosophila melanogaster                 41\_43 19577/33/1/0 (0\%/0\%/0\%)\\
Echinops telfairi                       40\_1 14309/8/9348/0 (0\%/65\%/0\%)\\
Gallus gallus                           41\_1p 20667/13/455/0 (0\%/2\%/0\%)\\
Gasterosteus aculeatus                  41\_1a 27181/13/138/0 (0\%/0\%/0\%)\\
Homo sapiens                            41\_36c 47004/41/6/0 (0\%/0\%/0\%)\\
Loxodonta africana                      40\_1 14366/10/9618/0 (0\%/66\%/0\%)\\
Macaca mulatta                          41\_10a 36446/14/491/0 (0\%/1\%/0\%)\\
Monodelphis domestica                   41\_3a 30358/0/80/0 (0\%/0\%/0\%)\\
Mus musculus                            41\_36b 29026/34/2/0 (0\%/0\%/0\%)\\
Oryctolagus cuniculus                   41\_1a 13705/4/8615/0 (0\%/62\%/0\%)\\
Oryzias latipes                         41\_1 25880/0/546/0 (0\%/2\%/0\%)\\
Pan troglodytes                         41\_21 32667/4/739/0 (0\%/2\%/0\%)\\
Rattus norvegicus                       41\_34k 32996/34/686/0 (0\%/2\%/0\%)\\
Saccharomyces cerevisiae                41\_1d 4767/2/0/0 (0\%/0\%/0\%)\\
Takifugu rubripes                       41\_4c 22102/0/283/0 (0\%/1\%/0\%)\\
Tetraodon nigroviridis                  41\_1g 15841/1/225/0 (0\%/1\%/0\%)\\
Xenopus tropicalis                      41\_41b 28324/0/626/0 (0\%/2\%/0\%)\\
\\
\\
Software: M. Gouy \& M. Jacobzone\\
Data maintenance: L. Duret \& S. Penel\\
\\
Laboratoire de Biometrie et Biologie Evolutive\\
UMR CNRS 5558, Universite Claude Bernard - Lyon 1\\
43, bd du 11 Novembre 1918 F-69622 Villeurbanne Cedex\\


\subsection{Type names}
There are no subsequence type in this database
\section{ homolensnucl }
\subsection{Bank details}
HOMOLENS 3 Homologous genes from Ensembl 41 Last Updated: Jan 19, 2007\\
32,635,729,329 bases; 81,903 sequences; 5,717,782 subseqs; 0 refers.\\
\\
Aedes aegypti                           41\_1a 11360/0/2/0 (0\%/0\%/0\%)\\
Anopheles gambiae                       41\_3d 13510/0/31/0 (0\%/0\%/0\%)\\
Apis mellifera                          38\_2d 27755/1/269/0 (0\%/0\%/0\%)\\
Bos taurus                              41\_2 32556/7/620/12 (0\%/1\%/0\%)\\
Caenorhabditis elegans                  41\_160 25218/1/0/0 (0\%/0\%/0\%)\\
Canis familiaris                        41\_1j 29813/0/0/0 (0\%/0\%/0\%)\\
Caenorhabditis briggsae                 25 14712/0/23/1 (0\%/0\%/0\%)\\
Ciona intestinalis                      41\_2c 20000/0/128/0 (0\%/0\%/0\%)\\
Ciona savignyi                          41\_2b 20150/1/27/0 (0\%/0\%/0\%)\\
Danio rerio                             41\_6b 36065/5/361/0 (0\%/1\%/0\%)\\
Dasypus novemcinctus                    40\_1 13567/12/8857/0 (0\%/65\%/0\%)\\
Drosophila melanogaster                 41\_43 19577/33/1/0 (0\%/0\%/0\%)\\
Echinops telfairi                       40\_1 14309/8/9348/0 (0\%/65\%/0\%)\\
Gallus gallus                           41\_1p 20667/13/455/0 (0\%/2\%/0\%)\\
Gasterosteus aculeatus                  41\_1a 27181/13/138/0 (0\%/0\%/0\%)\\
Homo sapiens                            41\_36c 47004/41/6/0 (0\%/0\%/0\%)\\
Loxodonta africana                      40\_1 14366/10/9618/0 (0\%/66\%/0\%)\\
Macaca mulatta                          41\_10a 36446/14/491/0 (0\%/1\%/0\%)\\
Monodelphis domestica                   41\_3a 30358/0/80/0 (0\%/0\%/0\%)\\
Mus musculus                            41\_36b 29026/34/2/0 (0\%/0\%/0\%)\\
Oryctolagus cuniculus                   41\_1a 13705/4/8615/0 (0\%/62\%/0\%)\\
Oryzias latipes                         41\_1 25880/0/546/0 (0\%/2\%/0\%)\\
Pan troglodytes                         41\_21 32667/4/739/0 (0\%/2\%/0\%)\\
Rattus norvegicus                       41\_34k 32996/34/686/0 (0\%/2\%/0\%)\\
Saccharomyces cerevisiae                41\_1d 4767/2/0/0 (0\%/0\%/0\%)\\
Takifugu rubripes                       41\_4c 22102/0/283/0 (0\%/1\%/0\%)\\
Tetraodon nigroviridis                  41\_1g 15841/1/225/0 (0\%/1\%/0\%)\\
Xenopus tropicalis                      41\_41b 28324/0/626/0 (0\%/2\%/0\%)\\
\\
\\
Software by M. Gouy \& M. Jacobzone, Laboratoire de biometrie, Universite Lyon I

\subsection{Type names}
\noindent\begin{tabular}{ll}
\hline \hline
name & description\\
\hline
3'INT & .3I 3'intron \\
3'NCR & .3F  3'-non coding region \\
5'INT & .5I 5'intron \\
5'NCR & .5F  5'-non coding region \\
CDS & .PE protein coding region \\
ID & EMBL sequence data library entry \\
INT\_INT & .IN  internal intron \\
MISC\_RNA & .RN other structural RNA coding region \\
RRNA & .RR ribosomal RNA coding region \\
SCRNA & .SC small cytoplasmic RNA coding region \\
SNRNA & .SN small nuclear RNA coding region \\
TRNA & .TR transfer RNA coding region \\
\hline \hline
\end{tabular}

\section{ greview }
\subsection{Bank details}
EBI Genome Reviews. Acnuc Release 7. Last Updated: Feb 26, 2007\\
1,454,629,225 bases; 757 sequences; 2,619,601 subseqs; 346 refers.\\
385 organisms\\
Software by M. Gouy \& M. Jacobzone, Laboratoire de biometrie, Universite Lyon I

\subsection{Type names}
\noindent\begin{tabular}{ll}
\hline \hline
name & description\\
\hline
3'INT & .3I 3'intron \\
3'NCR & .3F  3'-non coding region \\
5'INT & .5I 5'intron \\
5'NCR & .5F  5'-non coding region \\
CDS & .PE protein coding region \\
ID & EMBL sequence data library entry \\
INT\_INT & .IN  internal intron \\
MISC\_RNA & .RN other structural RNA coding region \\
RRNA & .RR ribosomal RNA coding region \\
SCRNA & .SC small cytoplasmic RNA coding region \\
SNRNA & .SN small nuclear RNA coding region \\
TRNA & .TR transfer RNA coding region \\
\hline \hline
\end{tabular}

\section{ emglib }
\subsection{Bank details}
EMGLib Release 5 (December 9, 2003)\\
434,648,385 bases; 174 sequences; 413,521 subseqs; 169 refers.\\
Data compiled from various sources by Guy Perriere

\subsection{Type names}
\noindent\begin{tabular}{ll}
\hline \hline
name & description\\
\hline
CDS & .PE protein coding region \\
LOCUS & sequenced DNA fragment \\
MISC\_RNA & .RN other structural RNA coding region \\
RRNA & .RR ribosomal RNA coding region \\
SCRNA & .SC small cytoplasmic RNA coding region \\
SNRNA & .SN small nuclear RNA coding region \\
TRNA & .TR transfer RNA coding region \\
\hline \hline
\end{tabular}

\section{ HAMAPnucl }
\subsection{Bank details}
HAMAP - Acnuc Release - Nucleotides - Last Updated: Jun  5, 2007\\
1,645,184,936 bases; 12,518 sequences; 1,529,575 subseqs; 11,742 refers.\\
	\\
Microbian genomes\\
\\
Nucleotide data from  EMBL cross-references		\\
in protein data from  HAMAP Database\\
(http://us.expasy.org/sprot/hamap/)\\
developed by the Swiss-Prot group at the Swiss Institut of Bioinformatics\\
\\
\\
Software: M. Gouy \& M. Jacobzone\\
Data maintenance: L. Duret \& S. Penel\\
\\
Laboratoire de Biometrie et Biologie Evolutive\\
UMR CNRS 5558, Universite Claude Bernard - Lyon 1\\
43, bd du 11 Novembre 1918 F-69622 Villeurbanne Cedex\\


\subsection{Type names}
\noindent\begin{tabular}{ll}
\hline \hline
name & description\\
\hline
3'INT & .3I 3'intron \\
3'NCR & .3F  3'-non coding region \\
5'INT & .5I 5'intron \\
5'NCR & .5F  5'-non coding region \\
CDS & .PE protein coding region \\
ID & EMBL sequence data library entry \\
INT\_INT & .IN  internal intron \\
MISC\_RNA & .RN other structural RNA coding region \\
RRNA & .RR ribosomal RNA coding region \\
SCRNA & .SC small cytoplasmic RNA coding region \\
SNRNA & .SN small nuclear RNA coding region \\
TRNA & .TR transfer RNA coding region \\
\hline \hline
\end{tabular}

\section{ HAMAPprot }
\subsection{Bank details}
HAMAP - Acnuc Release -Proteins- Last Updated: Jun  5, 2007\\
59,598,968 amino acids; 189,746 sequences; 10,068 references.\\
\\
Microbian genomes\\
\\
Protein data from  HAMAP Database\\
(http://us.expasy.org/sprot/hamap/)\\
developed by the Swiss-Prot group at the Swiss Institut of Bioinformatics\\
\\
Software: M. Gouy \& M. Jacobzone\\
Data maintenance: L. Duret \& S. Penel\\
\\
Laboratoire de Biometrie et Biologie Evolutive\\
UMR CNRS 5558, Universite Claude Bernard - Lyon 1\\
43, bd du 11 Novembre 1918 F-69622 Villeurbanne Cedex\\


\subsection{Type names}
There are no subsequence type in this database
\section{ hoppsigen }
\subsection{Bank details}
there is no information available about the contents of this bank

\subsection{Type names}
\noindent\begin{tabular}{ll}
\hline \hline
name & description\\
\hline
ID & EMBL sequence data library entry \\
CDS & .PE protein coding region \\
TRNA & .TR transfer RNA coding region \\
RRNA & .RR ribosomal RNA coding region \\
MISC\_RNA & .RN other structural RNA coding region \\
SCRNA & .SC small cytoplasmic RNA coding region \\
SNRNA & .SN small nuclear RNA coding region \\
CDE & .PS \\
PPGENE & .PP \\
3'FL & .3F \\
5'FL & .5F \\
DIRECT\_REPEAT & .DR \\
REPEAT\_REGION & .RR \\
POLYA\_REGION & .PA \\
FL\_REPEAT & .FR \\
\hline \hline
\end{tabular}

\section{ nurebnucl }
\subsection{Bank details}
Nurebase 4.0 (26 September 2003) Last Updated: NOV 27, 2003\\
2,356,663 bases; 664 sequences; 518 subseqs; 787 refers.\\
Software by M. Gouy \& M. Jacobzone, Laboratoire de biometrie, Universite Lyon I

\subsection{Type names}
\noindent\begin{tabular}{ll}
\hline \hline
name & description\\
\hline
CDS & .PE protein coding region \\
ID & EMBL sequence data library entry \\
MISC\_RNA & .RN other structural RNA coding region \\
RRNA & .RR ribosomal RNA coding region \\
SCRNA & .SC small cytoplasmic RNA coding region \\
SNRNA & .SN small nuclear RNA coding region \\
TRNA & .TR transfer RNA coding region \\
\hline \hline
\end{tabular}

\section{ nurebprot }
\subsection{Bank details}
Nurebase 4.0 (26 September 2003) Last Updated: NOV 27, 2003\\
277,024 amino acids; 525 sequences; 634 references.\\
Software by M. Gouy \& M. Jacobzone, Laboratoire de biometrie, Universite Lyon I

\subsection{Type names}
There are no subsequence type in this database
\section{ taxobacgen }
\subsection{Bank details}
TaxoBacGen Rel. 7 (September 2005)\\
1,151,149,763 bases; 254,335 sequences; 847,767 subseqs; 63,879 refers.\\
Data compiled from GenBank by Gregory Devulder\\
Laboratoire de Biometrie \& Biologie Evolutive, Univ Lyon I\\
------------------------------\\
This database is a taxonomic genomic database.\\
It results from an expertise crossing the data nomenclature database DSMZ\\
\[http://www.dsmz.de/species/bacteria.htm Deutsche Sammlung von\\
Mikroorganismen und Zellkulturen GmbH, Braunschweig, Germany\]\\
and GenBank.\\
- Only contains sequences described under species present in\\
Bacterial Nomenclature Up-to-date.\\
- Names of species and genus validly published according to the\\
Bacteriological Code (names with standing in nomenclature) is\\
added in field "DEFINITION".\\
- A keyword "type strain" is added in field "FEATURES/source/strain" in\\
GenBank format definition to easyly identify Type Strain.\\
Taxobacgen is a genomic database designed for studies based on a strict\\
respect of up-to-date nomenclature and taxonomy.

\subsection{Type names}
\noindent\begin{tabular}{ll}
\hline \hline
name & description\\
\hline
CDS & .PE protein coding region \\
LOCUS & sequenced DNA fragment \\
MISC\_RNA & .RN other structural RNA coding region \\
RRNA & .RR ribosomal RNA coding region \\
SCRNA & .SC small cytoplasmic RNA coding region \\
SNRNA & .SN small nuclear RNA coding region \\
TRNA & .TR transfer RNA coding region \\
\hline \hline
\end{tabular}\end{Schunk}


\section{Session Informations}

This part was compiled under the following \Rlogo{}~environment:

\begin{itemize}
  \item R version 2.5.1 (2007-06-27), \verb|i386-apple-darwin8.9.1|
  \item Locale: \verb|C|
  \item Base packages: base, datasets, grDevices, graphics, methods,
    stats, utils
  \item Other packages: MASS~7.2-34, ade4~1.4-3, ape~1.9-4,
    gee~4.13-12, lattice~0.15-11, nlme~3.1-83, seqinr~1.1-1,
    xtable~1.4-3
\end{itemize}

% END - DO NOT REMOVE THIS LINE

%%%%%%%%%%%%  BIBLIOGRAPHY %%%%%%%%%%%%%%%%%
\clearpage
\addcontentsline{toc}{section}{References}
\bibliographystyle{plain}
\bibliography{../config/book}
\end{document}
