\documentclass{article}
\usepackage{graphicx} % Extended graphics inclusions
\usepackage{float}
\usepackage{url} % For \url{}
\usepackage{../config/atxy} % For front cover
\usepackage{amsfonts} % Needed for some fonts
\usepackage[usenames]{color} % Needed for colored R input/output
\usepackage{pdfcolmk} % Correct some problems with the color stack


\title{Informations about databases available at pbil}
\author{Lobry, J.R.}

\usepackage{/Library/Frameworks/R.framework/Resources/share/texmf/Sweave}
\begin{document}
%
% To change the R input/output style:
%
\definecolor{Soutput}{rgb}{0,0,0.56}
\definecolor{Sinput}{rgb}{0.56,0,0}
\DefineVerbatimEnvironment{Sinput}{Verbatim}
{formatcom={\color{Sinput}},fontsize=\footnotesize, baselinestretch=0.75}
\DefineVerbatimEnvironment{Soutput}{Verbatim}
{formatcom={\color{Soutput}},fontsize=\footnotesize, baselinestretch=0.75}
%
% Rlogo
%
\newcommand{\Rlogo}{\protect\includegraphics[height=1.8ex,keepaspectratio]{../figs/Rlogo.pdf}}
%
% Shortcut for seqinR:
%
\newcommand{\seqinr}{\texttt{seqin\bf{R}}}
\newcommand{\Seqinr}{\texttt{Seqin\bf{R}}}
\fvset{fontsize= \scriptsize}
%
% R output options and libraries to be loaded.
%
%
%  Sweave Options
%
% Put all figures in the fig folder and start the name with current file name.
% Do not produce EPS files
%


\maketitle
% BEGIN - DO NOT REMOVE THIS LINE
\label{subsequence}

\section{Introduction}
This section was compiled on \today. The list of available database at pbil 
(\url{http://pbil.univ-lyon1.fr/}) was:

\begin{Schunk}
\begin{Sinput}
 bankDefault <- choosebank()
 bankTP <- choosebank(tagbank = "TP")
 bankDEV <- choosebank(tagbank = "DEV")
 (banknames <- c(bankDefault, bankTP, bankDEV))
\end{Sinput}
\begin{Soutput}
 [1] "genbank"      "embl"         "emblwgs"      "swissprot"   
 [5] "ensembl"      "refseq"       "hobacnucl"    "hobacprot"   
 [9] "hovernucl"    "hoverprot"    "hogennucl"    "hogenprot"   
[13] "hogen4nucl"   "hogen4prot"   "homolensprot" "homolensnucl"
[17] "greview"      "HAMAPnucl"    "HAMAPprot"    "hoppsigen"   
[21] "nurebnucl"    "nurebprot"    "taxobacgen"   "emblTP"      
[25] "swissprotTP"  "hoverprotTP"  "hovernuclTP"  "emglib"      
[29] "trypano"      "ensembl41"    "ensembl34"    "genomicro1"  
[33] "genomicro2"   "microbes"     "macaca"       "canis"       
[37] "mouse38"      "homo46"      
\end{Soutput}
\end{Schunk}

This \LaTeX~file was automatically generated by the following \Rlogo{}~code:

\begin{Schunk}
\begin{Sinput}
 for (b in banknames) {
     cat(paste("\\section{", b, "}"), sep = "\n")
     openTry <- try(choosebank(b))
     if (inherits(openTry, "try-error")) {
         cat("There was a problem while trying to open this bank.\n")
         next
     }
     bankdetails <- sapply(banknameSocket$details, stresc, 
         USE.NAMES = FALSE)
     cat("\\subsection{Bank details}", sep = "\n")
     cat(bankdetails, sep = "\\\\\n")
     cat("\n")
     cat("\\subsection{Type names}", sep = "\n")
     types <- getType()
     if (is.null(nrow(types))) {
         cat("There are no subsequence type in this database", 
             sep = "\n")
     }
     else {
         cat("\\noindent\\begin{tabular}{llr}", sep = "\n")
         cat("\\hline \\hline", sep = "\n")
         cat("name & description & count \\\\", sep = "\n")
         cat("\\hline", sep = "\n")
         sumnelem <- 0
         for (i in 1:nrow(types)) {
             querytry <- try(query("mylist", paste("T=", types[i, 
                 "sname"]), virtual = TRUE))
             if (inherits(querytry, "try-error")) {
                 nelem <- 0
             }
             else {
                 nelem <- mylist$nelem
             }
             sumnelem <- sumnelem + nelem
             cat(paste(stresc(types[i, "sname"]), " & ", stresc(types[i, 
                 "libel"]), " & ", formatC(nelem, big.mark = ",", 
                 format = "d"), "\\\\"), sep = "\n")
         }
         cat("\\hline", sep = "\n")
         cat(paste(" & Total: &", formatC(sumnelem, big.mark = ",", 
             format = "d"), "\\\\"), sep = "\n")
         cat("\\hline \\hline", sep = "\n")
         cat("\\end{tabular}", sep = "\n")
         cat("\n")
     }
     closebank()
 }
\end{Sinput}
\section{ genbank }
\subsection{Bank details}
             ****     ACNUC Data Base Content      ****                         \\
         GenBank Rel. 162 (15 October 2007) Last Updated: Nov  7, 2007\\
81,938,639,153 bases; 77,983,785 sequences; 4,560,520 subseqs; 482,575 refers.\\
Software by M. Gouy, Lab. Biometrie et Biologie Evolutive, Universite Lyon I 

\subsection{Type names}
\noindent\begin{tabular}{llr}
\hline \hline
name & description & count \\
\hline
CDS  &  .PE protein coding region  &  4,884,506 \\
LOCUS  &  sequenced DNA fragment  &  75,664,837 \\
MISC\_RNA  &  .RN other structural RNA coding region  &  525,754 \\
RRNA  &  .RR mature ribosomal RNA  &  1,160,620 \\
SCRNA  &  .SC small cytoplasmic RNA  &  161 \\
SNRNA  &  .SN small nuclear RNA  &  452 \\
TRNA  &  .TR mature transfer RNA  &  307,975 \\
\hline
 & Total: & 82,544,305 \\
\hline \hline
\end{tabular}

\section{ embl }
\subsection{Bank details}
             ****     ACNUC Data Base Content      ****                         \\
      EMBL Library Release 92 (September 2007) Last Updated: Nov  7, 2007\\
99,074,185,873 bases; 80,771,301 sequences; 11,262,120 subseqs; 478,786 refers.\\
Software by M. Gouy, Laboratoire de biometrie, Universite Lyon I 

\subsection{Type names}
\noindent\begin{tabular}{llr}
\hline \hline
name & description & count \\
\hline
CDS  &  .PE protein coding region  &  11,630,177 \\
ID  &  Locus entry  &  78,403,512 \\
MISC\_RNA  &  .RN other structural RNA coding region  &  523,679 \\
RRNA  &  .RR Ribosomal RNA coding gene  &  1,162,292 \\
SCRNA  &  .SC small cytoplasmic RNA  &  5 \\
SNRNA  &  .SN small nuclear RNA  &  79 \\
TRNA  &  .TR Transfer RNA coding gene  &  313,677 \\
\hline
 & Total: & 92,033,421 \\
\hline \hline
\end{tabular}

\section{ emblwgs }
\subsection{Bank details}
             ****     ACNUC Data Base Content      ****                         \\
        EMBL Whole Genome Shotgun sequences Release 92  (September 2007)  \\
102,118,953,790 bases; 25,405,475 sequences; 1,087,882 subseqs; 377 refers.\\
Retrieval software by M. Gouy, Biometrie et Biologie Evolutive, Univ Lyon I. 

\subsection{Type names}
\noindent\begin{tabular}{llr}
\hline \hline
name & description & count \\
\hline
CDS  &  .PE protein coding region  &  1,071,882 \\
ID  &  EMBL sequence data library entry  &  25,404,989 \\
MISC\_RNA  &  .RN other structural RNA coding region  &  483 \\
RRNA  &  .RR ribosomal RNA coding region  &  1,696 \\
SCRNA  &  .SC small cytoplasmic RNA coding region  &  4 \\
SNRNA  &  .SN small nuclear RNA coding region  &  100 \\
TRNA  &  .TR transfer RNA coding region  &  14,203 \\
\hline
 & Total: & 26,493,357 \\
\hline \hline
\end{tabular}

\section{ swissprot }
\subsection{Bank details}
               ****     ACNUC Data Base Content      ****                       \\
    UniProt Rel. 12 (SWISS-PROT 54 + TrEMBL 37): Last Updated: Nov  5, 2007\\
          1,723,945,386 amino acids; 5,275,429 sequences; 298,590 references.\\
          Non-redundant compilation of SWISS-PROT + TrEMBL\\
Software by M. Gouy \& L. Duret, Laboratoire de biometrie, Universite Lyon I.

\subsection{Type names}
There are no subsequence type in this database
\section{ ensembl }
\subsection{Bank details}
            ****     ACNUC Data Base Content      ****   \\
                 Ensembl databases release 41        \\
Espece                                  Release/\#CDS(1)/STOP(2)/N(3)/miss(4)\\
Aedes aegypti                           41\_1a 11360/0/2/0 (0\%/0\%/0\%)\\
Anopheles gambiae                       41\_3d 13510/0/31/0 (0\%/0\%/0\%)\\
Apis mellifera                          38\_2d 27755/1/269/0 (0\%/0\%/0\%)\\
Bos taurus                              41\_2 32556/7/620/12 (0\%/1\%/0\%)\\
Caenorhabditis elegans                  41\_160 25218/1/0/0 (0\%/0\%/0\%)\\
Canis familiaris                        41\_1j 29813/0/0/0 (0\%/0\%/0\%)\\
Caenorhabditis briggsae                 25 14712/0/23/1 (0\%/0\%/0\%)\\
Ciona intestinalis                      41\_2c 20000/0/128/0 (0\%/0\%/0\%)\\
Ciona savignyi                          41\_2b 20150/1/27/0 (0\%/0\%/0\%)\\
Danio rerio                             41\_6b 36065/5/361/0 (0\%/1\%/0\%)\\
Dasypus novemcinctus                    40\_1 13567/12/8857/0 (0\%/65\%/0\%)\\
Drosophila melanogaster                 41\_43 19577/33/1/0 (0\%/0\%/0\%)\\
Echinops telfairi                       40\_1 14309/8/9348/0 (0\%/65\%/0\%)\\
Gallus gallus                           41\_1p 20667/13/455/0 (0\%/2\%/0\%)\\
Gasterosteus aculeatus                  41\_1a 27181/13/138/0 (0\%/0\%/0\%)\\
Homo sapiens                            41\_36c 47004/41/6/0 (0\%/0\%/0\%)\\
Loxodonta africana                      40\_1 14366/10/9618/0 (0\%/66\%/0\%)\\
Macaca mulatta                          41\_10a 36446/14/491/0 (0\%/1\%/0\%)\\
Monodelphis domestica                   41\_3a 30358/0/80/0 (0\%/0\%/0\%)\\
Mus musculus                            41\_36b 29026/34/2/0 (0\%/0\%/0\%)\\
Oryctolagus cuniculus                   41\_1a 13705/4/8615/0 (0\%/62\%/0\%)\\
Oryzias latipes                         41\_1 25880/0/546/0 (0\%/2\%/0\%)\\
Pan troglodytes                         41\_21 32667/4/739/0 (0\%/2\%/0\%)\\
Rattus norvegicus                       41\_34k 32996/34/686/0 (0\%/2\%/0\%)\\
Saccharomyces cerevisiae                41\_1d 4767/2/0/0 (0\%/0\%/0\%)

\subsection{Type names}
\noindent\begin{tabular}{llr}
\hline \hline
name & description & count \\
\hline
3'INT  &  .3I 3'intron  &  0 \\
3'NCR  &  .3F  3'-non coding region  &  188,480 \\
5'INT  &  .5I 5'intron  &  0 \\
5'NCR  &  .5F  5'-non coding region  &  485,916 \\
CDS  &  .PE protein coding region  &  659,922 \\
ID  &  EMBL sequence data library entry  &  742,978 \\
INT\_INT  &  .IN  internal intron  &  4,342,137 \\
MISC\_RNA  &  .RN other structural RNA coding region  &  54,036 \\
MRNA  &  .RN mRNA  &  659,922 \\
RRNA  &  .RR ribosomal RNA coding region  &  0 \\
SCRNA  &  .SC small cytoplasmic RNA coding region  &  0 \\
SNRNA  &  .SN small nuclear RNA coding region  &  0 \\
TRNA  &  .TR transfer RNA coding region  &  0 \\
\hline
 & Total: & 7,133,391 \\
\hline \hline
\end{tabular}

\section{ refseq }
\subsection{Bank details}
             ****     ACNUC Data Base Content      ****                         \\
               RefSeq 15.0 (1 January 2006) Last Updated: Jan 23, 2006\\
1,055,245,496 bases; 625,928 sequences; 254,162 subseqs; 205,831 refers.\\
Software by M. Gouy \& M. Jacobzone, Laboratoire de biometrie, Universite Lyon I 

\subsection{Type names}
\noindent\begin{tabular}{llr}
\hline \hline
name & description & count \\
\hline
3'INT  &  .3I 3'intron  &  0 \\
3'NCR  &  .3F  3'-non coding region  &  0 \\
5'INT  &  .5I 5'intron  &  0 \\
5'NCR  &  .5F  5'-non coding region  &  0 \\
CDS  &  .PE protein coding region  &  624,776 \\
INT\_INT  &  .IN  internal intron  &  0 \\
LOCUS  &  sequenced DNA fragment  &  255,273 \\
MISC\_RNA  &  .RN other structural RNA coding region  &  8 \\
RRNA  &  .RR ribosomal RNA coding region  &  0 \\
SCRNA  &  .SC small cytoplasmic RNA coding region  &  2 \\
SNRNA  &  .SN small nuclear RNA coding region  &  22 \\
TRNA  &  .TR transfer RNA coding region  &  9 \\
\hline
 & Total: & 880,090 \\
\hline \hline
\end{tabular}

\section{ hobacnucl }
\subsection{Bank details}
               ****     ACNUC Data Base Content      ****                      \\
          HOBACGEN - genomic data - Release 10 (February 12 2002)\\
432,023,804 bases; 168,814 sequences; 293,669 subseqs; 52,735 references.\\
                                                                               \\
              Bacteria + Archaea + Saccharomyces cerevisiae\\
            Genomic data from EMBL Release 69 (December 2001)\\


\subsection{Type names}
\noindent\begin{tabular}{llr}
\hline \hline
name & description & count \\
\hline
CDS  &  .PE protein coding region  &  306,455 \\
ID  &  EMBL sequence data library entry  &  94,694 \\
MISC\_RNA  &  .RN other structural RNA coding region  &  2,299 \\
RRNA  &  .RR ribosomal RNA coding region  &  51,562 \\
SCRNA  &  .SC small cytoplasmic RNA coding region  &  41 \\
SNRNA  &  .SN small nuclear RNA coding region  &  193 \\
TRNA  &  .TR transfer RNA coding region  &  7,239 \\
\hline
 & Total: & 462,483 \\
\hline \hline
\end{tabular}

\section{ hobacprot }
\subsection{Bank details}
               ****     ACNUC Data Base Content      ****\\
           HOBACGEN - protein data - Release 10 (February 12 2002)\\
          79,755,852 amino acids; 260,025 sequences; 37,383 references.\\
               \\
              Bacteria + Archaea + Saccharomyces cerevisiae\\
   Protein data from SWISS-PROT 40 + TrEMBL 19 + TrEMBL\_NEW: January 25, 2002\\
\\
Software: M. Gouy \& M. Jacobzone\\
Data maintenance: L. Duret \& G. Perriere\\
\\
Laboratoire de Biometrie et Biologie Evolutive\\
UMR CNRS 5558, Universite Claude Bernard - Lyon 1 \\
43, bd du 11 Novembre 1918 F-69622 Villeurbanne Cedex\\


\subsection{Type names}
There are no subsequence type in this database
\section{ hovernucl }
\subsection{Bank details}
               ****     ACNUC Data Base Content      ****                      \\
 HOVERGEN - genomic data - Release 48 (May 24 2007) Last Updated: May 24, 2007\\
2,500,248,516 bases; 541,405 sequences; 1,005,089 subseqs; 117,556 refers.\\
                                                                               \\
                       Vertebrate (chordata)\\
    Genomic data from EMBL Library Release 90 (March 2007)\\
\\
Retrieval software by M. Gouy \& M. Jacobzone, Lab. de Biometrie, UCB Lyon.\\
Data maintenance: L. Duret \& S. Penel\\


\subsection{Type names}
\noindent\begin{tabular}{llr}
\hline \hline
name & description & count \\
\hline
3'INT  &  .3I 3'intron  &  0 \\
3'NCR  &  .3F  3'-non coding region  &  129,921 \\
5'INT  &  .5I 5'intron  &  0 \\
5'NCR  &  .5F  5'-non coding region  &  120,642 \\
CDS  &  .PE protein coding region  &  613,473 \\
ID  &  EMBL sequence data library entry  &  371,759 \\
INT\_INT  &  .IN  internal intron  &  172,068 \\
MISC\_RNA  &  .RN other structural RNA coding region  &  249 \\
RRNA  &  .RR ribosomal RNA coding region  &  9,873 \\
SCRNA  &  .SC small cytoplasmic RNA coding region  &  24 \\
SNRNA  &  .SN small nuclear RNA coding region  &  55 \\
TRNA  &  .TR transfer RNA coding region  &  128,430 \\
\hline
 & Total: & 1,546,494 \\
\hline \hline
\end{tabular}

\section{ hoverprot }
\subsection{Bank details}
               ****     ACNUC Data Base Content      ****                       \\
 HOVERGEN - protein data - Release 48 (May 24 2007) Last Updated: May 24, 2007\\
          142,891,140 amino acids; 415,383 sequences; 114,560 references.\\
\\
                       Vertebrate (chordata)	  \\
    Protein data from UniProt Rel. 10 (SWISS-PROT 52 + TrEMBL 35) May 2007\\
\\
Software: M. Gouy \& M. Jacobzone\\
Data maintenance: L. Duret \& S. Penel\\
\\
Laboratoire de Biometrie et Biologie Evolutive\\
UMR CNRS 5558, Universite Claude Bernard - Lyon 1 \\
43, bd du 11 Novembre 1918 F-69622 Villeurbanne Cedex\\


\subsection{Type names}
There are no subsequence type in this database
\section{ hogennucl }
\subsection{Bank details}
               ****     ACNUC Data Base Content      ****                      \\
  HOGENOM - genomic data - Release 03 (Oct 14 2005) Last Updated: Nov  7, 2005\\
2,538,433,251 bases; 227,950 sequences; 4,136,134 subseqs; 82,281 refers.\\
                                                                               \\
                        Fully Sequenced Organisms\\
   Protein data from http://www.ebi.ac.uk/proteome/ (August, 2005)\\
          Genomic data from GenomeReview  (June 2005) \\
                  and  EMBL (June 2005)\\
	   ( 263 fully sequenced organisms)\\
\\
Retrieval software by M. Gouy \& M. Jacobzone, Lab. de Biometrie, UCB Lyon.\\
Data maintenance: L. Duret \& S. Penel\\
\\
Laboratoire de Biometrie et Biologie Evolutive\\
UMR CNRS 5558, Universite Claude Bernard - Lyon 1 \\
43, bd du 11 Novembre 1918 F-69622 Villeurbanne Cedex\\


\subsection{Type names}
\noindent\begin{tabular}{llr}
\hline \hline
name & description & count \\
\hline
ID  &  EMBL sequence data library entry  &  204,502 \\
CDS  &  .PE protein coding region  &  1,060,241 \\
TRNA  &  .TR transfer RNA coding region  &  49,216 \\
RRNA  &  .RR ribosomal RNA coding region  &  5,813 \\
MISC\_RNA  &  .RN other structural RNA coding region  &  861 \\
SCRNA  &  .SC small cytoplasmic RNA coding region  &  29 \\
SNRNA  &  .SN small nuclear RNA coding region  &  459 \\
3'INT  &  .3I 3'intron  &  309 \\
3'NCR  &  .3F  3'-non coding region  &  1,247,297 \\
5'INT  &  .5I 5'intron  &  1,263 \\
5'NCR  &  .5F  5'-non coding region  &  1,158,238 \\
INT\_INT  &  .IN  internal intron  &  635,856 \\
\hline
 & Total: & 4,364,084 \\
\hline \hline
\end{tabular}

\section{ hogenprot }
\subsection{Bank details}
               ****     ACNUC Data Base Content      ****                      \\
  HOGENOM - protein data - Release 03 (Oct 14 2005) Last Updated: Mar 10, 2006\\
          339,891,443 amino acids; 950,216 sequences; 92,805 references.\\
                                                                               \\
                        Fully Sequenced Organisms\\
   Protein data from http://www.ebi.ac.uk/proteome/ (August 2005)\\
	            ( 263 fully sequenced organisms)\\
\\
Retrieval software by M. Gouy \& M. Jacobzone, Lab. de Biometrie, UCB Lyon.\\
Data maintenance: L. Duret \& S. Penel\\
\\
Laboratoire de Biometrie et Biologie Evolutive\\
UMR CNRS 5558, Universite Claude Bernard - Lyon 1 \\
43, bd du 11 Novembre 1918 F-69622 Villeurbanne Cedex\\


\subsection{Type names}
There are no subsequence type in this database
\section{ hogen4nucl }
\subsection{Bank details}
               ****     ACNUC Data Base Content      ****                      \\
 HOGENOM - genomic data - Release 04 (Sept 18,2007) Last Updated: Oct  2, 2007\\
14,692,834,718 bases; 134,844 sequences; 7,862,206 subseqs; 512 refers.\\
                                                                               \\
                        Fully Sequenced Organisms\\
   				Genomes \\
	  511 fully sequenced organisms (eukarya, bacteria, archaea)\\
\\
Retrieval software by M. Gouy \& M. Jacobzone, Lab. de Biometrie, UCB Lyon.\\
Data maintenance: L. Duret \& S. Penel\\
\\
Laboratoire de Biometrie et Biologie Evolutive\\
UMR CNRS 5558, Universite Claude Bernard - Lyon 1 \\
43, bd du 11 Novembre 1918 F-69622 Villeurbanne Cedex\\


\subsection{Type names}
\noindent\begin{tabular}{llr}
\hline \hline
name & description & count \\
\hline
3'INT  &  .3I 3'intron  &  0 \\
3'NCR  &  .3F  3'-non coding region  &  1,476,296 \\
5'INT  &  .5I 5'intron  &  0 \\
5'NCR  &  .5F  5'-non coding region  &  1,720,484 \\
CDS  &  .PE protein coding region  &  2,125,031 \\
ID  &  EMBL sequence data library entry  &  54,323 \\
INT\_INT  &  .IN  internal intron  &  2,560,918 \\
MISC\_RNA  &  .RN other structural RNA coding region  &  22,520 \\
RRNA  &  .RR ribosomal RNA coding region  &  6,378 \\
SCRNA  &  .SC small cytoplasmic RNA coding region  &  11 \\
SNRNA  &  .SN small nuclear RNA coding region  &  231 \\
TRNA  &  .TR transfer RNA coding region  &  30,858 \\
\hline
 & Total: & 7,997,050 \\
\hline \hline
\end{tabular}

\section{ hogen4prot }
\subsection{Bank details}
               ****     ACNUC Data Base Content      ****                      \\
 HOGENOM - protein data - Release 04 (Sept 18,2007) Last Updated: Nov  6, 2007\\
          755,031,736 amino acids; 2,142,639 sequences; 0 references.\\
                                                                               \\
                        Fully Sequenced Organisms\\
   				Protein data \\
	  511 fully sequenced organisms (eukarya, bacteria, archaea)\\
\\
Retrieval software by M. Gouy \& M. Jacobzone, Lab. de Biometrie, UCB Lyon.\\
Data maintenance: L. Duret \& S. Penel\\
\\
Laboratoire de Biometrie et Biologie Evolutive\\
UMR CNRS 5558, Universite Claude Bernard - Lyon 1 \\
43, bd du 11 Novembre 1918 F-69622 Villeurbanne Cedex\\


\subsection{Type names}
There are no subsequence type in this database
\section{ homolensprot }
\subsection{Bank details}
               ****     ACNUC Data Base Content      ****                      \\
      HOMOLENS 3 - Homologous genes from Ensembl Last Updated: Jan 19, 2007\\
          224,528,520 amino acids; 474,339 sequences; 0 references.\\
	 \\
                        Ensembl 41 Organisms Translated CDS\\
Aedes aegypti                           41\_1a 11360/0/2/0 (0\%/0\%/0\%)\\
Anopheles gambiae                       41\_3d 13510/0/31/0 (0\%/0\%/0\%)\\
Apis mellifera                          38\_2d 27755/1/269/0 (0\%/0\%/0\%)\\
Bos taurus                              41\_2 32556/7/620/12 (0\%/1\%/0\%)\\
Caenorhabditis elegans                  41\_160 25218/1/0/0 (0\%/0\%/0\%)\\
Canis familiaris                        41\_1j 29813/0/0/0 (0\%/0\%/0\%)\\
Caenorhabditis briggsae                 25 14712/0/23/1 (0\%/0\%/0\%)\\
Ciona intestinalis                      41\_2c 20000/0/128/0 (0\%/0\%/0\%)\\
Ciona savignyi                          41\_2b 20150/1/27/0 (0\%/0\%/0\%)\\
Danio rerio                             41\_6b 36065/5/361/0 (0\%/1\%/0\%)\\
Dasypus novemcinctus                    40\_1 13567/12/8857/0 (0\%/65\%/0\%)\\
Drosophila melanogaster                 41\_43 19577/33/1/0 (0\%/0\%/0\%)\\
Echinops telfairi                       40\_1 14309/8/9348/0 (0\%/65\%/0\%)\\
Gallus gallus                           41\_1p 20667/13/455/0 (0\%/2\%/0\%)\\
Gasterosteus aculeatus                  41\_1a 27181/13/138/0 (0\%/0\%/0\%)\\
Homo sapiens                            41\_36c 47004/41/6/0 (0\%/0\%/0\%)\\
Loxodonta africana                      40\_1 14366/10/9618/0 (0\%/66\%/0\%)\\
Macaca mulatta                          41\_10a 36446/14/491/0 (0\%/1\%/0\%)\\
Monodelphis domestica                   41\_3a 30358/0/80/0 (0\%/0\%/0\%)\\
Mus musculus                            41\_36b 29026/34/2/0 (0\%/0\%/0\%)\\
Oryctolagus cuniculus                   41\_1a 13705/4/8615/0 (0\%/62\%/0\%)\\
Oryzias latipes                         41\_1 25880/0/546/0 (0\%/2\%/0\%)\\
Pan troglodytes                         41\_21 32667/4/739/0 (0\%/2\%/0\%)\\
Rattus norvegicus                       41\_34k 32996/34/686/0 (0\%/2\%/0\%)\\
Saccharomyces cerevisiae                41\_1d 4767/2/0/0 (0\%/0\%/0\%)\\
Takifugu rubripes                       41\_4c 22102/0/283/0 (0\%/1\%/0\%)\\
Tetraodon nigroviridis                  41\_1g 15841/1/225/0 (0\%/1\%/0\%)\\
Xenopus tropicalis                      41\_41b 28324/0/626/0 (0\%/2\%/0\%)\\
 \\
	     \\
Software: M. Gouy \& M. Jacobzone\\
Data maintenance: L. Duret \& S. Penel\\
\\
Laboratoire de Biometrie et Biologie Evolutive\\
UMR CNRS 5558, Universite Claude Bernard - Lyon 1 \\
43, bd du 11 Novembre 1918 F-69622 Villeurbanne Cedex\\


\subsection{Type names}
There are no subsequence type in this database
\section{ homolensnucl }
\subsection{Bank details}
            ****     ACNUC Data Base Content      ****  \\
       HOMOLENS 3 Homologous genes from Ensembl 41 Last Updated: Jan 19, 2007\\
32,635,729,329 bases; 81,903 sequences; 5,717,782 subseqs; 0 refers.\\
\\
Aedes aegypti                           41\_1a 11360/0/2/0 (0\%/0\%/0\%)\\
Anopheles gambiae                       41\_3d 13510/0/31/0 (0\%/0\%/0\%)\\
Apis mellifera                          38\_2d 27755/1/269/0 (0\%/0\%/0\%)\\
Bos taurus                              41\_2 32556/7/620/12 (0\%/1\%/0\%)\\
Caenorhabditis elegans                  41\_160 25218/1/0/0 (0\%/0\%/0\%)\\
Canis familiaris                        41\_1j 29813/0/0/0 (0\%/0\%/0\%)\\
Caenorhabditis briggsae                 25 14712/0/23/1 (0\%/0\%/0\%)\\
Ciona intestinalis                      41\_2c 20000/0/128/0 (0\%/0\%/0\%)\\
Ciona savignyi                          41\_2b 20150/1/27/0 (0\%/0\%/0\%)\\
Danio rerio                             41\_6b 36065/5/361/0 (0\%/1\%/0\%)\\
Dasypus novemcinctus                    40\_1 13567/12/8857/0 (0\%/65\%/0\%)\\
Drosophila melanogaster                 41\_43 19577/33/1/0 (0\%/0\%/0\%)\\
Echinops telfairi                       40\_1 14309/8/9348/0 (0\%/65\%/0\%)\\
Gallus gallus                           41\_1p 20667/13/455/0 (0\%/2\%/0\%)\\
Gasterosteus aculeatus                  41\_1a 27181/13/138/0 (0\%/0\%/0\%)\\
Homo sapiens                            41\_36c 47004/41/6/0 (0\%/0\%/0\%)\\
Loxodonta africana                      40\_1 14366/10/9618/0 (0\%/66\%/0\%)\\
Macaca mulatta                          41\_10a 36446/14/491/0 (0\%/1\%/0\%)\\
Monodelphis domestica                   41\_3a 30358/0/80/0 (0\%/0\%/0\%)\\
Mus musculus                            41\_36b 29026/34/2/0 (0\%/0\%/0\%)\\
Oryctolagus cuniculus                   41\_1a 13705/4/8615/0 (0\%/62\%/0\%)\\
Oryzias latipes                         41\_1 25880/0/546/0 (0\%/2\%/0\%)\\
Pan troglodytes                         41\_21 32667/4/739/0 (0\%/2\%/0\%)\\
Rattus norvegicus                       41\_34k 32996/34/686/0 (0\%/2\%/0\%)\\
Saccharomyces cerevisiae                41\_1d 4767/2/0/0 (0\%/0\%/0\%)\\
Takifugu rubripes                       41\_4c 22102/0/283/0 (0\%/1\%/0\%)\\
Tetraodon nigroviridis                  41\_1g 15841/1/225/0 (0\%/1\%/0\%)\\
Xenopus tropicalis                      41\_41b 28324/0/626/0 (0\%/2\%/0\%)\\
\\
	     \\
Software by M. Gouy \& M. Jacobzone, Laboratoire de biometrie, Universite Lyon I 

\subsection{Type names}
\noindent\begin{tabular}{llr}
\hline \hline
name & description & count \\
\hline
3'INT  &  .3I 3'intron  &  0 \\
3'NCR  &  .3F  3'-non coding region  &  188,371 \\
5'INT  &  .5I 5'intron  &  0 \\
5'NCR  &  .5F  5'-non coding region  &  485,692 \\
CDS  &  .PE protein coding region  &  659,680 \\
ID  &  EMBL sequence data library entry  &  81,903 \\
INT\_INT  &  .IN  internal intron  &  4,339,670 \\
MISC\_RNA  &  .RN other structural RNA coding region  &  44,369 \\
RRNA  &  .RR ribosomal RNA coding region  &  0 \\
SCRNA  &  .SC small cytoplasmic RNA coding region  &  0 \\
SNRNA  &  .SN small nuclear RNA coding region  &  0 \\
TRNA  &  .TR transfer RNA coding region  &  0 \\
\hline
 & Total: & 5,799,685 \\
\hline \hline
\end{tabular}

\section{ greview }
\subsection{Bank details}
             ****     ACNUC Data Base Content      ****                         \\
        EBI Genome Reviews. Acnuc Release 7. Last Updated: Feb 26, 2007\\
1,454,629,225 bases; 757 sequences; 2,619,601 subseqs; 346 refers.\\
385 organisms\\
Software by M. Gouy \& M. Jacobzone, Laboratoire de biometrie, Universite Lyon I 

\subsection{Type names}
\noindent\begin{tabular}{llr}
\hline \hline
name & description & count \\
\hline
3'INT  &  .3I 3'intron  &  0 \\
3'NCR  &  .3F  3'-non coding region  &  619,385 \\
5'INT  &  .5I 5'intron  &  0 \\
5'NCR  &  .5F  5'-non coding region  &  647,404 \\
CDS  &  .PE protein coding region  &  1,233,635 \\
ID  &  EMBL sequence data library entry  &  757 \\
INT\_INT  &  .IN  internal intron  &  92,062 \\
MISC\_RNA  &  .RN other structural RNA coding region  &  0 \\
RRNA  &  .RR ribosomal RNA coding region  &  4,288 \\
SCRNA  &  .SC small cytoplasmic RNA coding region  &  11 \\
SNRNA  &  .SN small nuclear RNA coding region  &  54 \\
TRNA  &  .TR transfer RNA coding region  &  22,762 \\
\hline
 & Total: & 2,620,358 \\
\hline \hline
\end{tabular}

\section{ HAMAPnucl }
\subsection{Bank details}
               ****     ACNUC Data Base Content      ****                       \\
        HAMAP - Acnuc Release - Nucleotides - Last Updated: Jun  5, 2007\\
1,645,184,936 bases; 12,518 sequences; 1,529,575 subseqs; 11,742 refers.\\
    	  \\
	                   Microbian genomes\\
			   \\
                Nucleotide data from  EMBL cross-references		\\
                  in protein data from  HAMAP Database\\
			(http://us.expasy.org/sprot/hamap/)\\
developed by the Swiss-Prot group at the Swiss Institut of Bioinformatics\\
 \\
 \\
Software: M. Gouy \& M. Jacobzone\\
Data maintenance: L. Duret \& S. Penel\\
 \\
Laboratoire de Biometrie et Biologie Evolutive\\
UMR CNRS 5558, Universite Claude Bernard - Lyon 1 \\
43, bd du 11 Novembre 1918 F-69622 Villeurbanne Cedex\\


\subsection{Type names}
\noindent\begin{tabular}{llr}
\hline \hline
name & description & count \\
\hline
3'INT  &  .3I 3'intron  &  0 \\
3'NCR  &  .3F  3'-non coding region  &  0 \\
5'INT  &  .5I 5'intron  &  0 \\
5'NCR  &  .5F  5'-non coding region  &  0 \\
CDS  &  .PE protein coding region  &  1,494,504 \\
ID  &  EMBL sequence data library entry  &  10,986 \\
INT\_INT  &  .IN  internal intron  &  0 \\
MISC\_RNA  &  .RN other structural RNA coding region  &  1,496 \\
RRNA  &  .RR ribosomal RNA coding region  &  5,925 \\
SCRNA  &  .SC small cytoplasmic RNA coding region  &  12 \\
SNRNA  &  .SN small nuclear RNA coding region  &  46 \\
TRNA  &  .TR transfer RNA coding region  &  29,124 \\
\hline
 & Total: & 1,542,093 \\
\hline \hline
\end{tabular}

\section{ HAMAPprot }
\subsection{Bank details}
               ****     ACNUC Data Base Content      ****                       \\
          HAMAP - Acnuc Release -Proteins- Last Updated: Jun  5, 2007\\
          59,598,968 amino acids; 189,746 sequences; 10,068 references.\\
	  \\
	                   Microbian genomes\\
			 \\
                    Protein data from  HAMAP Database\\
	          (http://us.expasy.org/sprot/hamap/)\\
developed by the Swiss-Prot group at the Swiss Institut of Bioinformatics\\
  \\
Software: M. Gouy \& M. Jacobzone\\
Data maintenance: L. Duret \& S. Penel\\
 \\
Laboratoire de Biometrie et Biologie Evolutive\\
UMR CNRS 5558, Universite Claude Bernard - Lyon 1 \\
43, bd du 11 Novembre 1918 F-69622 Villeurbanne Cedex\\
 

\subsection{Type names}
There are no subsequence type in this database
\section{ hoppsigen }
\subsection{Bank details}
NA

\subsection{Type names}
\noindent\begin{tabular}{llr}
\hline \hline
name & description & count \\
\hline
ID  &  EMBL sequence data library entry  &  9,757 \\
CDS  &  .PE protein coding region  &  3,814 \\
TRNA  &  .TR transfer RNA coding region  &  0 \\
RRNA  &  .RR ribosomal RNA coding region  &  0 \\
MISC\_RNA  &  .RN other structural RNA coding region  &  0 \\
SCRNA  &  .SC small cytoplasmic RNA coding region  &  0 \\
SNRNA  &  .SN small nuclear RNA coding region  &  0 \\
CDE  &  .PS  &  9,757 \\
PPGENE  &  .PP  &  9,757 \\
3'FL  &  .3F  &  3,656 \\
5'FL  &  .5F  &  730 \\
DIRECT\_REPEAT  &  .DR  &  15,592 \\
REPEAT\_REGION  &  .RR  &  133,215 \\
POLYA\_REGION  &  .PA  &  1,694 \\
FL\_REPEAT  &  .FR  &  0 \\
\hline
 & Total: & 187,972 \\
\hline \hline
\end{tabular}

\section{ nurebnucl }
\subsection{Bank details}
             ****     ACNUC Data Base Content      ****                         \\
         Nurebase 4.0 (26 September 2003) Last Updated: NOV 27, 2003\\
2,356,663 bases; 664 sequences; 518 subseqs; 787 refers.\\
Software by M. Gouy \& M. Jacobzone, Laboratoire de biometrie, Universite Lyon I 

\subsection{Type names}
\noindent\begin{tabular}{llr}
\hline \hline
name & description & count \\
\hline
CDS  &  .PE protein coding region  &  767 \\
ID  &  EMBL sequence data library entry  &  415 \\
MISC\_RNA  &  .RN other structural RNA coding region  &  0 \\
RRNA  &  .RR ribosomal RNA coding region  &  0 \\
SCRNA  &  .SC small cytoplasmic RNA coding region  &  0 \\
SNRNA  &  .SN small nuclear RNA coding region  &  0 \\
TRNA  &  .TR transfer RNA coding region  &  0 \\
\hline
 & Total: & 1,182 \\
\hline \hline
\end{tabular}

\section{ nurebprot }
\subsection{Bank details}
             ****     ACNUC Data Base Content      ****                         \\
         Nurebase 4.0 (26 September 2003) Last Updated: NOV 27, 2003\\
          277,024 amino acids; 525 sequences; 634 references.\\
Software by M. Gouy \& M. Jacobzone, Laboratoire de biometrie, Universite Lyon I 

\subsection{Type names}
There are no subsequence type in this database
\section{ taxobacgen }
\subsection{Bank details}
               ****     ACNUC Data Base Content      ****\\
                 TaxoBacGen Rel. 7 (September 2005)\\
1,151,149,763 bases; 254,335 sequences; 847,767 subseqs; 63,879 refers.\\
	Data compiled from GenBank by Gregory Devulder \\
        Laboratoire de Biometrie \& Biologie Evolutive, Univ Lyon I\\
------------------------------\\
This database is a taxonomic genomic database. \\
It results from an expertise crossing the data nomenclature database DSMZ\\
\[http://www.dsmz.de/species/bacteria.htm Deutsche Sammlung von\\
Mikroorganismen und Zellkulturen GmbH, Braunschweig, Germany\]\\
and GenBank. \\
- Only contains sequences described under species present in \\
Bacterial Nomenclature Up-to-date.\\
- Names of species and genus validly published according to the\\
Bacteriological Code (names with standing in nomenclature) is \\
added in field "DEFINITION".\\
- A keyword "type strain" is added in field "FEATURES/source/strain" in\\
GenBank format definition to easyly identify Type Strain.\\
Taxobacgen is a genomic database designed for studies based on a strict\\
respect of up-to-date nomenclature and taxonomy.

\subsection{Type names}
\noindent\begin{tabular}{llr}
\hline \hline
name & description & count \\
\hline
CDS  &  .PE protein coding region  &  879,340 \\
LOCUS  &  sequenced DNA fragment  &  168,243 \\
MISC\_RNA  &  .RN other structural RNA coding region  &  3,720 \\
RRNA  &  .RR ribosomal RNA coding region  &  34,965 \\
SCRNA  &  .SC small cytoplasmic RNA coding region  &  36 \\
SNRNA  &  .SN small nuclear RNA coding region  &  0 \\
TRNA  &  .TR transfer RNA coding region  &  15,798 \\
\hline
 & Total: & 1,102,102 \\
\hline \hline
\end{tabular}

\section{ emblTP }
\subsection{Bank details}
             ****     ACNUC Data Base Content      ****                         \\
              EMBL Library Release 78 WITHOUT ESTs  (March 2004)\\
27,571,397,913 bases; 12,533,594 sequences; 1,604,500 subseqs; 339,186 refers.\\
Software by M. Gouy \& M. Jacobzone, Laboratoire de biometrie, Universite Lyon I 

\subsection{Type names}
\noindent\begin{tabular}{llr}
\hline \hline
name & description & count \\
\hline
CDS  &  .PE protein coding region  &  1,746,728 \\
ID  &  Locus entry  &  11,856,048 \\
MISC\_RNA  &  .RN other structural RNA coding region  &  109,101 \\
RRNA  &  .RR Ribosomal RNA coding gene  &  320,935 \\
SCRNA  &  .SC small cytoplasmic RNA  &  311 \\
SNRNA  &  .SN small nuclear RNA  &  1,687 \\
TRNA  &  .TR Transfer RNA coding gene  &  103,284 \\
\hline
 & Total: & 14,138,094 \\
\hline \hline
\end{tabular}

\section{ swissprotTP }
\subsection{Bank details}
               ****     ACNUC Data Base Content      ****                       \\
  UniProt Rel. 1 (SWISS-PROT 43 + TrEMBL 26 + NEW): Last Updated: May  3, 2004\\
          459,974,342 amino acids; 1,451,384 sequences; 200,578 references.\\
          Non-redundant compilation of SWISS-PROT + TrEMBL (minus  data  \\
                     integrated  into  SWISS-PROT)\\
Software by M. Gouy \& L. Duret, Laboratoire de biometrie, Universite Lyon I.

\subsection{Type names}
There are no subsequence type in this database
\section{ hoverprotTP }
\subsection{Bank details}
               ****     ACNUC Data Base Content      ****                       \\
         HOVERGEN - Release 45 (Jan 22 2004) Last Updated: Jan 22, 2004\\
          77,617,436 amino acids; 227,047 sequences; 85,918 references.\\
\\
                       Vertebrate (chordata)	  \\
Protein data from SWISS-PROT Rel. 42  + TrEMBL Rel. 25 + TrEMBL\_NEW: Dec 1, 2003\\
\\
Software: M. Gouy \& M. Jacobzone\\
Data maintenance: L. Duret \& S. Penel\\
\\
Laboratoire de Biometrie et Biologie Evolutive\\
UMR CNRS 5558, Universite Claude Bernard - Lyon 1 \\
43, bd du 11 Novembre 1918 F-69622 Villeurbanne Cedex\\


\subsection{Type names}
There are no subsequence type in this database
\section{ hovernuclTP }
\subsection{Bank details}
               ****     ACNUC Data Base Content      ****                      \\
         HOVERGEN - Release 45 (Jan 22 2004) Last Updated: Jan 22, 2004\\
844,876,418 bases; 300,108 sequences; 757,209 subseqs; 97,608 refers.\\
                                                                               \\
                       Vertebrate (chordata)\\
             Genomic data from EMBL Release 77  (December 2003)\\
\\
Retrieval software by M. Gouy \& M. Jacobzone, Lab. de Biometrie, UCB Lyon.

\subsection{Type names}
\noindent\begin{tabular}{llr}
\hline \hline
name & description & count \\
\hline
3'INT  &  .3I 3'intron  &  535 \\
3'NCR  &  .3F  3'-non coding region  &  170,566 \\
5'INT  &  .5I 5'intron  &  1,381 \\
5'NCR  &  .5F  5'-non coding region  &  159,238 \\
CDS  &  .PE protein coding region  &  274,599 \\
ID  &  EMBL sequence data library entry  &  210,301 \\
INT\_INT  &  .IN  internal intron  &  132,033 \\
MISC\_RNA  &  .RN other structural RNA coding region  &  164 \\
RRNA  &  .RR ribosomal RNA coding region  &  2,426 \\
SCRNA  &  .SC small cytoplasmic RNA coding region  &  9 \\
SNRNA  &  .SN small nuclear RNA coding region  &  41 \\
TRNA  &  .TR transfer RNA coding region  &  35,309 \\
\hline
 & Total: & 986,602 \\
\hline \hline
\end{tabular}

\section{ emglib }
\subsection{Bank details}
               ****     ACNUC Database Content      ****\\
                  EMGLib Release 5 (December 9, 2003)\\
434,648,385 bases; 174 sequences; 413,521 subseqs; 169 refers.\\
           Data compiled from various sources by Guy Perriere

\subsection{Type names}
\noindent\begin{tabular}{llr}
\hline \hline
name & description & count \\
\hline
CDS  &  .PE protein coding region  &  404,721 \\
LOCUS  &  sequenced DNA fragment  &  174 \\
MISC\_RNA  &  .RN other structural RNA coding region  &  239 \\
RRNA  &  .RR ribosomal RNA coding region  &  1,409 \\
SCRNA  &  .SC small cytoplasmic RNA coding region  &  8 \\
SNRNA  &  .SN small nuclear RNA coding region  &  6 \\
TRNA  &  .TR transfer RNA coding region  &  7,138 \\
\hline
 & Total: & 413,695 \\
\hline \hline
\end{tabular}

\section{ trypano }
\subsection{Bank details}
             ****     ACNUC Data Base Content      ****                         \\
         trypano Rel. 1 (27 Janvier 2004) Last Updated: Jan 27, 2004\\
117,177,046 bases; 158,838 sequences; 4,744 subseqs; 2,114 refers.\\
	Genomic data from GenBank Rel. 139 (15 December 2003)\\
Software by M. Gouy \& M. Jacobzone, Laboratoire de biometrie, Universite Lyon I 

\subsection{Type names}
\noindent\begin{tabular}{llr}
\hline \hline
name & description & count \\
\hline
LOCUS  &  sequenced DNA fragment  &  157,983 \\
CDS  &  .PE protein coding region  &  5,137 \\
TRNA  &  .TR transfer RNA coding region  &  38 \\
RRNA  &  .RR ribosomal RNA coding region  &  206 \\
MISC\_RNA  &  .RN other structural RNA coding region  &  192 \\
SCRNA  &  .SC small cytoplasmic RNA coding region  &  0 \\
SNRNA  &  .SN small nuclear RNA coding region  &  26 \\
\hline
 & Total: & 163,582 \\
\hline \hline
\end{tabular}

\section{ ensembl41 }
\subsection{Bank details}
            ****     ACNUC Data Base Content      ****   \\
                 Ensembl databases release 41        \\
Espece                                  Release/\#CDS(1)/STOP(2)/N(3)/miss(4)\\
Aedes aegypti                           41\_1a 11360/0/2/0 (0\%/0\%/0\%)\\
Anopheles gambiae                       41\_3d 13510/0/31/0 (0\%/0\%/0\%)\\
Apis mellifera                          38\_2d 27755/1/269/0 (0\%/0\%/0\%)\\
Bos taurus                              41\_2 32556/7/620/12 (0\%/1\%/0\%)\\
Caenorhabditis elegans                  41\_160 25218/1/0/0 (0\%/0\%/0\%)\\
Canis familiaris                        41\_1j 29813/0/0/0 (0\%/0\%/0\%)\\
Caenorhabditis briggsae                 25 14712/0/23/1 (0\%/0\%/0\%)\\
Ciona intestinalis                      41\_2c 20000/0/128/0 (0\%/0\%/0\%)\\
Ciona savignyi                          41\_2b 20150/1/27/0 (0\%/0\%/0\%)\\
Danio rerio                             41\_6b 36065/5/361/0 (0\%/1\%/0\%)\\
Dasypus novemcinctus                    40\_1 13567/12/8857/0 (0\%/65\%/0\%)\\
Drosophila melanogaster                 41\_43 19577/33/1/0 (0\%/0\%/0\%)\\
Echinops telfairi                       40\_1 14309/8/9348/0 (0\%/65\%/0\%)\\
Gallus gallus                           41\_1p 20667/13/455/0 (0\%/2\%/0\%)\\
Gasterosteus aculeatus                  41\_1a 27181/13/138/0 (0\%/0\%/0\%)\\
Homo sapiens                            41\_36c 47004/41/6/0 (0\%/0\%/0\%)\\
Loxodonta africana                      40\_1 14366/10/9618/0 (0\%/66\%/0\%)\\
Macaca mulatta                          41\_10a 36446/14/491/0 (0\%/1\%/0\%)\\
Monodelphis domestica                   41\_3a 30358/0/80/0 (0\%/0\%/0\%)\\
Mus musculus                            41\_36b 29026/34/2/0 (0\%/0\%/0\%)\\
Oryctolagus cuniculus                   41\_1a 13705/4/8615/0 (0\%/62\%/0\%)\\
Oryzias latipes                         41\_1 25880/0/546/0 (0\%/2\%/0\%)\\
Pan troglodytes                         41\_21 32667/4/739/0 (0\%/2\%/0\%)\\
Rattus norvegicus                       41\_34k 32996/34/686/0 (0\%/2\%/0\%)\\
Saccharomyces cerevisiae                41\_1d 4767/2/0/0 (0\%/0\%/0\%)

\subsection{Type names}
\noindent\begin{tabular}{llr}
\hline \hline
name & description & count \\
\hline
3'INT  &  .3I 3'intron  &  0 \\
3'NCR  &  .3F  3'-non coding region  &  188,480 \\
5'INT  &  .5I 5'intron  &  0 \\
5'NCR  &  .5F  5'-non coding region  &  485,916 \\
CDS  &  .PE protein coding region  &  659,922 \\
ID  &  EMBL sequence data library entry  &  742,978 \\
INT\_INT  &  .IN  internal intron  &  4,342,137 \\
MISC\_RNA  &  .RN other structural RNA coding region  &  54,036 \\
MRNA  &  .RN mRNA  &  659,922 \\
RRNA  &  .RR ribosomal RNA coding region  &  0 \\
SCRNA  &  .SC small cytoplasmic RNA coding region  &  0 \\
SNRNA  &  .SN small nuclear RNA coding region  &  0 \\
TRNA  &  .TR transfer RNA coding region  &  0 \\
\hline
 & Total: & 7,133,391 \\
\hline \hline
\end{tabular}

\section{ ensembl34 }
\subsection{Bank details}
            ****     ACNUC Data Base Content      ****   \\
                 Ensembl databases release 34        \\
Espece                                   \#CDS(1)/STOP(2)/N(3)/miss(4)\\
Apis mellifera                          27736/1/269 (0\%/0\%/0\%)\\
Caenorhabditis briggsae                 14712/0/23 (0\%/0\%/0\%)\\
Caenorhabditis elegans                  25797/1/0 (0\%/0\%/0\%)\\
Gallus gallus                           28392/20/298 (0\%/1\%/0\%)\\
Pan troglodytes                         39538/6129/770 (15\%/1\%/0\%)\\
Ciona intestinalis                      21574/0/58 (0\%/0\%/0\%)\\
Bos taurus                              32647/7/617 (0\%/1\%/0\%)\\
Canis familiaris                        29998/0/0 (0\%/0\%/1\%)\\
Drosophila melanogaser                  19350/18/1 (0\%/0\%/0\%)\\
Fugu rubripes                           22099/0/283 (0\%/1\%/0\%)\\
Homo sapiens                            36919/48/24 (0\%/0\%/2\%)\\
Macaca mulatta                          31370/94/8360 (0\%/26\%/0\%)\\
Anopheles gambiae                       15799/0/19 (0\%/0\%/0\%)\\
Mus musculus                            35075/36/60 (0\%/0\%/1\%)\\
Monodelphis domestica                   13249/0/59 (0\%/0\%/0\%)\\
Rattus norvegicus                       32241/25/607 (0\%/1\%/2\%)\\
Tetraodon nigroviridis                  16275/1/233 (0\%/1\%/0\%)\\
Xenopus tropicalis                      52684/1/906 (0\%/1\%/0\%)\\
Saccharomyces cerevisiae                6680/22/0 (0\%/0\%/0\%)\\
Danio rerio                             32109/0/281 (0\%/0\%/0\%)\\
1:\# of CDS;2:CDS with internal stop;3:CDS with undetermined codon;4:missing CDS\\
   warning : cds located on contigs were removed\\
29,605,509,937 bases; 368,619 sequences; 5,302,323 subseqs; 0 refers.\\
Software by M. Gouy \& M. Jacobzone, Laboratoire de biometrie, Universite Lyon I 

\subsection{Type names}
\noindent\begin{tabular}{llr}
\hline \hline
name & description & count \\
\hline
3'INT  &  .3I 3'intron  &  0 \\
3'NCR  &  .3F  3'-non coding region  &  156,270 \\
5'INT  &  .5I 5'intron  &  0 \\
5'NCR  &  .5F  5'-non coding region  &  280,705 \\
CDS  &  .PE protein coding region  &  534,246 \\
ID  &  EMBL sequence data library entry  &  368,619 \\
INT\_INT  &  .IN  internal intron  &  3,763,554 \\
MISC\_RNA  &  .RN other structural RNA coding region  &  33,511 \\
MRNA  &  .RN mRNA  &  534,037 \\
RRNA  &  .RR ribosomal RNA coding region  &  0 \\
SCRNA  &  .SC small cytoplasmic RNA coding region  &  0 \\
SNRNA  &  .SN small nuclear RNA coding region  &  0 \\
TRNA  &  .TR transfer RNA coding region  &  0 \\
\hline
 & Total: & 5,670,942 \\
\hline \hline
\end{tabular}

\section{ genomicro1 }
\subsection{Bank details}
             ****     ACNUC Data Base Content      ****                         \\
              Genomicro1 (15 June 2006) Last Updated: Jul  6, 2006\\
10,758,321,631 bases; 203,021 sequences; 1,870,190 subseqs; 0 refers.\\
\\
\\


\subsection{Type names}
\noindent\begin{tabular}{llr}
\hline \hline
name & description & count \\
\hline
3'NCR  &  .3F  3'-non coding region  &  34,406 \\
5'NCR  &  .5F  5'-non coding region  &  84,333 \\
CDS  &  .PE protein coding region  &  91,991 \\
EXON  &  .EX exon  &  545,221 \\
GENE  &  .GE gene  &  74,019 \\
ID  &  EMBL sequence data library entry  &  203,020 \\
INT\_INT  &  .IN  internal intron  &  531,587 \\
MISC\_FEATURE  &  .MF misc feature  &  397,178 \\
MISC\_RNA  &  .RN other structural RNA coding region  &  19,549 \\
MRNA  &  .MA mrna  &  91,907 \\
RRNA  &  .RR ribosomal RNA coding region  &  0 \\
SCRNA  &  .SC small cytoplasmic RNA coding region  &  0 \\
SNRNA  &  .SN small nuclear RNA coding region  &  0 \\
TRNA  &  .TR transfer RNA coding region  &  0 \\
\hline
 & Total: & 2,073,211 \\
\hline \hline
\end{tabular}

\section{ genomicro2 }
\subsection{Bank details}
             ****     ACNUC Data Base Content      ****                         \\
              Genomicro (15 June 2006) Last Updated: Jul  6, 2006\\
9,861,694,164 bases; 280 sequences; 2,858,384 subseqs; 0 refers.\\
\\
\\


\subsection{Type names}
\noindent\begin{tabular}{llr}
\hline \hline
name & description & count \\
\hline
3'NCR  &  .3F  3'-non coding region  &  45,675 \\
5'NCR  &  .5F  5'-non coding region  &  99,248 \\
CDS  &  .PE protein coding region  &  135,798 \\
EXON  &  .EX exon  &  831,047 \\
GENE  &  .GE gene  &  97,094 \\
ID  &  EMBL sequence data library entry  &  278 \\
INT\_INT  &  .IN  internal intron  &  1,037,668 \\
MISC\_FEATURE  &  .MF misc feature  &  457,916 \\
MISC\_RNA  &  .RN other structural RNA coding region  &  18,143 \\
MRNA  &  .MA mrna  &  135,797 \\
RRNA  &  .RR ribosomal RNA coding region  &  0 \\
SCRNA  &  .SC small cytoplasmic RNA coding region  &  0 \\
SNRNA  &  .SN small nuclear RNA coding region  &  0 \\
TRNA  &  .TR transfer RNA coding region  &  0 \\
\hline
 & Total: & 2,858,664 \\
\hline \hline
\end{tabular}

\section{ microbes }
\subsection{Bank details}
             ****     ACNUC Data Base Content      ****                         \\
       NCBI Microbial Genomes. Acnuc Release 1. Last Updated: Mar  6, 2007\\
1,628,177,183 bases; 826 sequences; 1,511,540 subseqs; 853 refers.\\
464 organisms\\
Software by M. Gouy \& M. Jacobzone, Laboratoire de biometrie, Universite Lyon I 

\subsection{Type names}
\noindent\begin{tabular}{llr}
\hline \hline
name & description & count \\
\hline
3'INT  &  .3I 3'intron  &  0 \\
3'NCR  &  .3F  3'-non coding region  &  0 \\
5'INT  &  .5I 5'intron  &  0 \\
5'NCR  &  .5F  5'-non coding region  &  0 \\
CDS  &  .PE protein coding region  &  1,478,713 \\
INT\_INT  &  .IN  internal intron  &  0 \\
LOCUS  &  sequenced DNA fragment  &  826 \\
MISC\_RNA  &  .RN other structural RNA coding region  &  1,407 \\
RRNA  &  .RR ribosomal RNA coding region  &  5,324 \\
SCRNA  &  .SC small cytoplasmic RNA coding region  &  6 \\
SNRNA  &  .SN small nuclear RNA coding region  &  6 \\
TRNA  &  .TR transfer RNA coding region  &  26,084 \\
\hline
 & Total: & 1,512,366 \\
\hline \hline
\end{tabular}

\section{ macaca }
\subsection{Bank details}
             ****     ACNUC Data Base Content      ****                         \\
        Ensembl Macaca mulatta (Rel. 45\_10e) Last Updated: Jun 25, 2007\\
3,028,222,859 bases; 94,529 sequences; 41,046 subseqs; 0 refers.\\
Software by M. Gouy, Laboratoire de biometrie, Universite Lyon I 

\subsection{Type names}
\noindent\begin{tabular}{llr}
\hline \hline
name & description & count \\
\hline
3'INT  &  .3I 3'intron  &  0 \\
3'NCR  &  .3F  3'-non coding region  &  0 \\
5'INT  &  .5I 5'intron  &  0 \\
5'NCR  &  .5F  5'-non coding region  &  0 \\
CDS  &  .PE protein coding region  &  35,875 \\
ID  &  EMBL sequence data library entry  &  94,529 \\
INT\_INT  &  .IN  internal intron  &  0 \\
MISC\_RNA  &  .RN other structural RNA coding region  &  5,171 \\
RRNA  &  .RR ribosomal RNA coding region  &  0 \\
SCRNA  &  .SC small cytoplasmic RNA coding region  &  0 \\
SNRNA  &  .SN small nuclear RNA coding region  &  0 \\
TRNA  &  .TR transfer RNA coding region  &  0 \\
\hline
 & Total: & 135,575 \\
\hline \hline
\end{tabular}

\section{ canis }
\subsection{Bank details}
             ****     ACNUC Data Base Content      ****                         \\
       Ensembl Canis familiaris  (Rel. 45\_2c) Last Updated: Jul  4, 2007\\
2,531,673,731 bases; 2,585 sequences; 29,227 subseqs; 0 refers.\\
Software by M. Gouy, Laboratoire de biometrie, Universite Lyon I 

\subsection{Type names}
\noindent\begin{tabular}{llr}
\hline \hline
name & description & count \\
\hline
3'INT  &  .3I 3'intron  &  0 \\
3'NCR  &  .3F  3'-non coding region  &  0 \\
5'INT  &  .5I 5'intron  &  0 \\
5'NCR  &  .5F  5'-non coding region  &  0 \\
CDS  &  .PE protein coding region  &  25,559 \\
ID  &  EMBL sequence data library entry  &  2,585 \\
INT\_INT  &  .IN  internal intron  &  0 \\
MISC\_RNA  &  .RN other structural RNA coding region  &  3,668 \\
RRNA  &  .RR ribosomal RNA coding region  &  0 \\
SCRNA  &  .SC small cytoplasmic RNA coding region  &  0 \\
SNRNA  &  .SN small nuclear RNA coding region  &  0 \\
TRNA  &  .TR transfer RNA coding region  &  0 \\
\hline
 & Total: & 31,812 \\
\hline \hline
\end{tabular}

\section{ mouse38 }
\subsection{Bank details}
             ****     ACNUC Data Base Content      ****                         \\
          Ensembl Mus musculus (Rel.38\_35) Last Updated: Jul  6, 2007\\
2,676,276,909 bases; 7,505 sequences; 35,002 subseqs; 0 refers.\\
Software by M. Gouy, Laboratoire de biometrie, Universite Lyon I 

\subsection{Type names}
\noindent\begin{tabular}{llr}
\hline \hline
name & description & count \\
\hline
3'INT  &  .3I 3'intron  &  0 \\
3'NCR  &  .3F  3'-non coding region  &  0 \\
5'INT  &  .5I 5'intron  &  0 \\
5'NCR  &  .5F  5'-non coding region  &  0 \\
CDS  &  .PE protein coding region  &  31,984 \\
ID  &  EMBL sequence data library entry  &  7,505 \\
INT\_INT  &  .IN  internal intron  &  0 \\
MISC\_RNA  &  .RN other structural RNA coding region  &  3,018 \\
RRNA  &  .RR ribosomal RNA coding region  &  0 \\
SCRNA  &  .SC small cytoplasmic RNA coding region  &  0 \\
SNRNA  &  .SN small nuclear RNA coding region  &  0 \\
TRNA  &  .TR transfer RNA coding region  &  0 \\
\hline
 & Total: & 42,507 \\
\hline \hline
\end{tabular}

\section{ homo46 }
\subsection{Bank details}
             ****     ACNUC Data Base Content      ****                         \\
    Ensembl Homo sapiens rel 46\_36h (September 2007) Last Updated: Sep  3, 2007\\
3,662,694,094 bases; 3,859 sequences; 551,227 subseqs; 0 refers.\\
MENU Nber of lines= 21                                                         

\subsection{Type names}
\noindent\begin{tabular}{llr}
\hline \hline
name & description & count \\
\hline
3'INT  &  .3I 3'intron  &  0 \\
3'NCR  &  .3F  3'-non coding region  &  21,970 \\
5'INT  &  .5I 5'intron  &  0 \\
5'NCR  &  .5F  5'-non coding region  &  44,246 \\
CDS  &  .PE protein coding region  &  50,736 \\
ID  &  EMBL sequence data library entry  &  3,859 \\
INT\_INT  &  .IN  internal intron  &  424,658 \\
MISC\_RNA  &  .RN other structural RNA coding region  &  9,617 \\
RRNA  &  .RR ribosomal RNA coding region  &  0 \\
SCRNA  &  .SC small cytoplasmic RNA coding region  &  0 \\
SNRNA  &  .SN small nuclear RNA coding region  &  0 \\
TRNA  &  .TR transfer RNA coding region  &  0 \\
\hline
 & Total: & 555,086 \\
\hline \hline
\end{tabular}\end{Schunk}


\section*{Session Informations}

This part was compiled under the following \Rlogo{}~environment:

\begin{itemize}
  \item R version 2.6.0 (2007-10-03), \verb|i386-apple-darwin8.10.1|
  \item Locale: \verb|C|
  \item Base packages: base, datasets, grDevices, graphics, methods,
    stats, utils
  \item Other packages: MASS~7.2-36, ade4~1.4-4, ape~2.0-1,
    gee~4.13-13, lattice~0.16-5, nlme~3.1-85, seqinr~1.1-3,
    xtable~1.5-1
  \item Loaded via a namespace (and not attached): grid~2.6.0,
    rcompgen~0.1-15
\end{itemize}
There were two compilation steps:

\begin{itemize}
  \item \Rlogo{} compilation time was: Wed Nov  7 12:40:01 2007
  \item \LaTeX{} compilation time was: \today
\end{itemize}

% END - DO NOT REMOVE THIS LINE

%%%%%%%%%%%%  BIBLIOGRAPHY %%%%%%%%%%%%%%%%%
\clearpage
\addcontentsline{toc}{section}{References}
\bibliographystyle{plain}
\bibliography{../config/book}
\end{document}
