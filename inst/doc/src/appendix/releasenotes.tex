\documentclass{article}
\usepackage{graphicx} % Extended graphics inclusions
\usepackage{float}
\usepackage{url} % For \url{}
\usepackage{../config/atxy} % For front cover
\usepackage{amsfonts} % Needed for some fonts
\usepackage[usenames]{color} % Needed for colored R input/output
\usepackage{pdfcolmk} % Correct some problems with the color stack


\title{Release notes}
\author{Lobry, J.R. \and Nec\c{s}ulea, A. \and Palmeira, L. \and Penel, S.}

\usepackage{/Library/Frameworks/R.framework/Resources/share/texmf/Sweave}
\begin{document}
%
% To change the R input/output style:
%
\definecolor{Soutput}{rgb}{0,0,0.56}
\definecolor{Sinput}{rgb}{0.56,0,0}
\DefineVerbatimEnvironment{Sinput}{Verbatim}
{formatcom={\color{Sinput}},fontsize=\footnotesize, baselinestretch=0.75}
\DefineVerbatimEnvironment{Soutput}{Verbatim}
{formatcom={\color{Soutput}},fontsize=\footnotesize, baselinestretch=0.75}
%
% Rlogo
%
\newcommand{\Rlogo}{\protect\includegraphics[height=1.8ex,keepaspectratio]{../figs/Rlogo.pdf}}
%
% Shortcut for seqinR:
%
\newcommand{\seqinr}{\texttt{seqin\bf{R}}}
\newcommand{\Seqinr}{\texttt{Seqin\bf{R}}}
\fvset{fontsize= \scriptsize}
%
% R output options and libraries to be loaded.
%
%
%  Sweave Options
%
% Put all figures in the fig folder and start the name with current file name.
% Do not produce EPS files
%


\maketitle
% BEGIN - DO NOT REMOVE THIS LINE

\section{release 1.0-7}

\begin{itemize}

\item A new \emph{experimental} function \texttt{extractseqs()} to download
sequences thru zlib compressed sockets from an ACNUC server is released. 
Preliminary tests suggest that working with about 100,000 CDS is possible with 
a home ADSL connection. See the manual for some \texttt{system.time()}
examples.

\item As pointed by e-mail on 16 Nov 2006 by Emmanuel Prestat the URL
used in \texttt{dia.bactgensize()} was no more available, this has been fixed
in the current version.

\item As pointed by e-mail on 16 Nov 2006 by Guy Perri{\`e}re, the
function \texttt{oriloc()} was no more compatible with glimmer\footnote{
Glimmer is a program to predict coding sequences in microbial genomes \cite{SalzbergSL1998, DelcherAL1999}.
} 3.0
outputs. The function has gained a new argument \texttt{glimmer.version}
defaulting to 3, but the value 2 is still functional for backward compatibility
with old glimmer outputs.

\item As pointed by e-mail on 24 Oct 2006 by Lionel Guy 
(\url{http://pbil.univ-lyon1.fr/seqinr/seqinrhtmlannuel/03/0089.html})
there was no default value for the \texttt{as.string} argument in
the \texttt{getSequence.SeqFastadna()}. A default \texttt{FALSE}
value is now present for backward compatibility with older code.

\item New utility vectorized function \texttt{stresc()}  to escape \LaTeX~special 
characters present in a string.

\item New low level function \texttt{readsmj()} available.

\item A new function \texttt{readfirstrec()} to get 
the record count of the specified ACNUC index file is now available.

\item Function \texttt{getType()} called without arguments will now use
the default ACNUC database to return available subsequence types.

\item Function \texttt{read.alignment()} now also accepts \texttt{file} in
addition to \texttt{File} as argument.

\item A new function \texttt{rearranged.oriloc()} is available. This
method, based on \texttt{oriloc()}, can be used to detect the effect of
the replication mechanism on DNA base composition asymmetry, in
prokaryotic chromosomes.

\item New function \texttt{extract.breakpoints()}, used to extract
breakpoints in rearranged nucleotide skews. This function uses the
\texttt{segmented} package to define the position of the breakpoints.

\item New function \texttt{draw.rearranged.oriloc()} available, to plot
nucleotide skews on artificially rearranged prokaryotic chromosomes.

\item New function \texttt{gbk2g2.euk()} available. Similarly to 
\texttt{gbk2g2()}, this function extracts the coding sequence annotations
from a GenBank format file. This function is specifically designed for
eukaryotic sequences, \textit{i.e.} with introns. The output file will contain
the coordinates of the exons, along with the name of the CDS to which
they belong.

\item After an e-mail by Marcelo Bertalan on 26 Mar 2007, a bug in
\texttt{oriloc()} when the \texttt{gbk} argument was \texttt{NULL}
was found and fixed by Anamaria Nec\c{s}ulea.

\item Functions \texttt{translate()} and \texttt{getTrans()} have gained
a new argument \texttt{NAstring} to represent untranslatable amino-
acids, defaulting to character "X".

\item There was a typo for the total number of printed bases in
the ACNUC books \cite{GautierC1982a, GautierC1982b} : 474,439 should be 526,506.

\item Function \texttt{invers()} has been deleted.

\item Functions \texttt{translate()}, \texttt{getTrans()} and \texttt{comp()} have gained a 
new argument \texttt{ambiguous} defaulting to FALSE allowing to handle 
ambiguous bases. If TRUE, ambiguous bases are taken into account so that 
for instance GGN is translated to Gly in the standard genetic code.

\item New function \texttt{amb()} to return the list of nucleotide matching 
a given IUPAC nucleotide symbol.

\item Function \texttt{count()} has gained a new argument \texttt{alphabet}
so that oligopeptides counts are now possible. Thanks to Gabriel Valiente
for this suggestion. The functions \texttt{zscore()}, \texttt{rho()} and
\texttt{summary.SeqFastadna()} have also an argument \texttt{alphabet} which
is forwarded to \texttt{count()}.

\end{itemize}

\section{release 1.0-6}

Release 1.0-6 is a minor release to fix a problem found and solved by Kurt Hornik
(namely a change from \texttt{SET\_ELEMENT} to \texttt{SET\_STRING\_ELT}
in C code for \texttt{s2c()} in file \texttt{util.c}). The few changes are
as follows.

\begin{itemize}

\item More typographical option for the output \LaTeX~table of \texttt{tablecode()}
are now available to outline deviations from the standard genetic code (see example in the
appendix "genetic codes" of the manual).

\item A new dataset \texttt{aaindex} extracted from the aaindex database
\cite{aaindex1, aaindex2, aaindex3} is now available. It contains a list
of 544 physicochemical and biological properties for the 20 amino-acids

\item The default value for argument \texttt{dia} is now \texttt{FALSE}
in function \texttt{tablecode()}.

\item The example code for \texttt{data(chargaff)} has been changed.

\end{itemize}
\section{release 1.0-5}

\begin{itemize}

\item A  new function \texttt{dotPlot()} is now available.

\item A new function \texttt{crelistfromclientdata()} is now available to
create a list on the server from a local file of sequence names, sequence
accession numbers, species names, or keywords names.

\item A new function \texttt{pmw()} to compute the molecular weight of
a protein is now available.

\item A new function \texttt{reverse.align()} contributed by Anamaria Nec\c{s}ulea
is now available to align CDS at the protein level and then reverse translate this at
the nucleic acid level from a \texttt{clustalw} output. This can be done on the fly
if \texttt{clustalw} is available on your platform.

\item An undocumented behavior was reported by Guy Perri{\`e}re for \texttt{uco()}
when computing RSCU on sequences where an amino-acid is missing. There is
now a new argument \texttt{NA.rscu} that allows the user to force the
missing values to his favorite magic value.

\item There was a bug in \texttt{read.fasta()}: some sequence names were
truncated, this is now fixed (thanks to Marcus G. Daniels for pointing this).
In order to be more consistent with standard functions such as \texttt{read.table()}
or \texttt{scan()}, the file argument starts now with a lower case letter (\texttt{file})
in function \texttt{read.fasta()}, but the old-style \texttt{File} is still
functional for forward-compatibility. There is a new logical argument in \texttt{read.fasta()}
named \texttt{as.string} to allow sequences to be returned as strings instead of
vector of single characters. The automatic conversion of DNA sequences into
lower case letters can now be disabled with the new logical argument
\texttt{forceDNAtolower}. It is also possible to disable the automatic attributes
settings with the new logical argument \texttt{set.attributes}.

\item A new function \texttt{write.fasta()} is now available.

\item The function \texttt{kaks()} now forces character in sequences to upper case.
This default behavior can be neutralized in order to save time by setting the 
argument \texttt{forceUpperCase} to \texttt{FALSE}.

\end{itemize}

\section{release 1.0-4}

\begin{itemize}
\item The scaling factor $n_{\bullet\bullet}$ was missing in equation \ref{chi}.
\item The files \texttt{louse.fasta}, \texttt{louse.names}, \texttt{gopher.fasta}, \texttt{gopher.names}
and \texttt{ortho.fasta} that were used for examples in the previous version of this document are
no more downloaded from the internet since they are now distributed in the \texttt{sequences/} folder
of the package.
\item An example of synonymous and non synonymous codon usage analysis was added to the
vignette along with two toy data sets (\texttt{toyaa} and \texttt{toycodon}).
\item A FAQ section was added to the vignette.
\item A bug in \texttt{getAnnot()} when the number of lines was zero is now fixed.
\item There is now a new argument, \texttt{latexfile}, in \texttt{tablecode()} to export genetic codes
tables in a \LaTeX~document, for instance table \ref{../tables/code3.tex} and table \ref{../tables/code4.tex} here.
\item There is now a new argument, \texttt{freq}, in \texttt{count()}
  to compute word frequencies instead of counts.
\item Function \texttt{splitseq()} has been entirely rewritten to improve speed.
\item Functions computing the G+C content: \texttt{GC(), GC1(), GC2(),
  GC3()} were rewritten to improve speed, and their document files
  were merged to facilitate usage.
\item The following new functions have been added:
\begin{itemize}
\item \texttt{syncodons()} returns all synonymous codons for a given
  codon. Argument \texttt{numcode} specifies the desired genetic code.
\item \texttt{ucoweight()} returns codon usage bias on a sequence as
  the number of synonymous codons present in the sequence for each
  amino acid.
\item \texttt{synsequence()} generates a random coding sequence which
      is synonymous to a given sequence and has a chosen codon usage
      bias.
\item \texttt{permutation()} generates a new sequence from a given
  sequence, while maintaining some constraints from the given sequence
  such as nucleotide frequency, codon usage bias, ...
\item \texttt{rho()} computes the rho statistic on dinucleotides as
  defined in \cite{Karlin}.
\item \texttt{zscore()} computes the zscore statistic on dinucleotides
  as defined in \cite{UV}.
\end{itemize}
\item Two datasets (\texttt{dinucl} and \texttt{prochlo}) were added
  to illustrate these new functions.

\end{itemize}

\section{release 1.0-3}

\begin{itemize}
\item The new package maintainer is Dr. Simon Penel, PhD, who has now a fixed position in the
laboratory that issued \seqinr~(\texttt{penel@biomserv.univ-lyon1.fr}). Delphine Charif was
successful too to get a fixed position in the same lab, with now a different research task (but who knows?).
Thanks to the close vicinity of our pioneering maintainers the transition was sweet. The DESCRIPTION
file of the \seqinr{} package has been updated to take this into account.

\item The reference paper for the package is now \textit{in press}. We do not have the full
reference for now, you may use \texttt{citation("seqinr")} to check if it is complete now:
\begin{Schunk}
\begin{Sinput}
 citation("seqinr")
\end{Sinput}
\begin{Soutput}
To cite seqinR in publications use:

  in the body of the text (J.R. Lobry, personal communication), or
  wait for the exact complete reference.

A BibTeX entry for LaTeX users is

  @incollection{,
    author = {D. Charif and J.R. Lobry},
    title = {SeqinR 1.0-2: a contributed package to the R project for statistical computing devoted to biological sequences retrieval and analysis.},
    booktitle = {Structural approaches to sequence evolution: Molecules, networks, populations},
    year = {2006},
    editor = {U. Bastolla, M. Porto, H.E. Roman and M. Vendruscolo},
    volume = {NA},
    series = {Biological and Medical Physics, Biomedical Engineering},
    pages = {NA},
    address = {New York},
    month = {NA},
    organization = {NA},
    publisher = {Springer Verlag},
    note = {in press},
  }

Note that the orginal article and updates are available in the
../../../../../seqinr.Rcheck/seqinr/doc/ folder in PDF format
\end{Soutput}
\end{Schunk}

\item There was a bug when sending a \texttt{gfrag} request to the server for long (Mb range) 
sequences. The length argument was converted to scientific notations that are not understand by the
server. This is now corrected and should work up the the Gb scale.

\item The \texttt{query()} function has been improved by de-looping list element info request,
there are now download at once which is much more efficient. For example, a query from a
researcher-home ADSL connection with a list with about 1000 elements was 60 seconds and
is now only 4 seconds (\textit{i.e.} 15 times faster now).

\item A new parameter \texttt{virtual} has been added to \texttt{query()} 
so that long lists can stay on the server without trying to download
them automatically. A query like \texttt{query(s\$socket,"allcds","t=cds", virtual = TRUE)} is 
now possible.

\item Relevant genetic codes and frames are now automatically propagated.

\item \Seqinr{}~sends now its name and version number to the server.

% Pas arrviÈ ‡ le repoduire
%
%\item A bug as been reported for intensive \texttt{kaks()} calls.

\item Strict control on ambiguous DNA base alphabet has been relaxed.

\item Default value for parameter \texttt{invisible} of function \texttt{query()} is now \texttt{TRUE}.

\end{itemize}



\section{Session Informations}

This part was compiled under the following \Rlogo{}~environment:

\begin{itemize}
  \item R version 2.4.1 (2006-12-18), \verb|i386-apple-darwin8.8.1|
  \item Locale: \verb|C|
  \item Base packages: base, datasets, grDevices, graphics, methods,
    stats, utils
  \item Other packages: MASS~7.2-30, ade4~1.4-2, ape~1.9-3,
    gee~4.13-12, lattice~0.14-16, nlme~3.1-78, seqinr~1.0-7,
    xtable~1.4-3
\end{itemize}

% END - DO NOT REMOVE THIS LINE


%%%%%%%%%%%%  BIBLIOGRAPHY %%%%%%%%%%%%%%%%%
\clearpage
\addcontentsline{toc}{section}{References}
\bibliographystyle{plain}
\bibliography{../config/book}
\end{document}




